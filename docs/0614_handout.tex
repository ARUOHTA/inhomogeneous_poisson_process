%%% Copyright: Template was made by Zurab Vashakidze for TAPDE Workshops %%%
\documentclass[a4paper,11pt]{article}
\usepackage{graphicx} % Required for inserting images
\usepackage[
	%top=1.5cm, % Top margin
        top=0.5cm, % Top margin
	bottom=1.5cm, % Bottom margin
	inner=1.5cm, % Inner margin
	outer=1.5cm, % Outer margin
	% footskip=1.4cm, % Space from the bottom margin to the baseline of the footer
	% headsep=0.8cm, % Space from the top margin to the baseline of the header
	% headheight=0.5cm, % Height of the header
	%showframe % Uncomment to show the frames around the margins for debugging purposes
]{geometry}
\usepackage{enumerate}
\usepackage[utf8]{inputenc} % Required for inputting international characters
\usepackage[T1]{fontenc} % Output font encoding for international characters
\usepackage[english]{babel}
\usepackage{amsmath,amsfonts,amssymb,amsthm,amstext}
\usepackage{microtype} % Improve typography
\usepackage{lmodern}
\usepackage{hyperref} % Required for links
\hypersetup{
	colorlinks=false,
	%urlcolor=\primarycolor, % Colour for \url and \href links
	%linkcolor=\primarycolor, % Colour for \nameref links
	hidelinks, % Hide the default boxes around links
}
\usepackage{nopageno}
\usepackage[nottoc]{tocbibind}

\usepackage{tikz}
\usepackage{makecell}
\usepackage[numbers,sort]{natbib}

\usepackage{graphicx}
\usepackage{lipsum}
\usepackage{amsmath}
\usepackage{amsthm}
\newtheorem{definition}{Definition}
\newtheorem{theorem}{Theorem}

\setlength{\bibsep}{1.5pt}

% \usepackage{titlesec}
% \titleformat*{\section}{\LARGE\sectionbf}
% 添加首行缩进,1个字符
\usepackage{indentfirst}
\setlength{\parindent}{1em}

\linespread{0.8}

% \setlengtha{\parskip}{0pt}
% \renewcommand{\bibname}{\fontsize{8}{8}\selectfont References}
% \patchcmd{\thebibliography}{\section*{\refname}}{}{}{}

\newcommand{\indep}{\mathop{\perp\!\!\!\!\perp}}
\newcommand{\notindep}{\mathop{\not \perp\!\!\!\!\perp}}

\title{Bayesian nonparametric methods for spatial multinomial count data}
%%% If there are multiple authors of the abstract, please indicate the presenting author by underlining their full name. %%%

% \author{First Author\textsuperscript{1}, \underline{Second Author}\textsuperscript{1,2}, Third Author\textsuperscript{2}, etc. \\
%     \footnotesize \textsuperscript{1} Department, University or Institution, City (Town), Country \\
%     \footnotesize \textsuperscript{2} Department, University or Institution, City (Town), Country \\
%     \footnotesize \textsuperscript{3} Department, University or Institution, City (Town), Country
% }
% \vspace{-10cm}
\author{Aru Ohta\\
\small{\textit{Graduate School of Informatics},
\textit{Kyoto University}}\\
\ \\
TSUMURA Hiroomi\\
\small{\textit{Faculty of Culture and Information Science},
\textit{Department of Culture and Information Science}}\\
\ \\
Hisayuki Hara\\
\small {\textit{Institute for Liberal Arts and Sciences},
\textit{Kyoto University}}}
\date{}

\begin{document}
% \vspace{-1.5cm}
\maketitle

\vspace{-1cm}
\section{Introduction}
\vspace{-0.25cm}
Spatial multinomial count data arise across diverse scientific domains, where observations at different locations follow multinomial distributions with location-dependent parameters. Such data structures are prevalent in ecological species composition studies, archaeological artifact distribution analyses, and epidemiological disease classification research. Traditional approaches often rely on parametric assumptions or simple kernel smoothing, which fail to capture complex spatial dependencies and uncertainty in the underlying stochastic processes.

The Kernel Stick-Breaking Process (KSBP) provides a flexible nonparametric Bayesian framework, but its application to spatial multinomial data remains limited. We propose a Multinomial Kernel Stick-Breaking Process (Multinomial-KSBP) that extends the KSBP framework to handle spatial multinomial count data with comprehensive uncertainty quantification for location-dependent multinomial parameters.

\vspace{-0.4cm}
\section{Multinomial Kernel Stick-Breaking Process}
\vspace{-0.25cm}
Consider a spatial domain $\mathcal{D} \subset \mathbb{R}^2$ and observation locations $s_1, \ldots, s_n \in \mathcal{D}$. At each location $s_i$, we observe multinomial count data $\mathbf{y}_i = (y_{i1}, \ldots, y_{iK})^\top \in \mathbb{N}^K$ with total count $N_i = \sum_{k=1}^K y_{ik}$. The objective is to estimate the multinomial probability vector $\boldsymbol{\pi}(s) = (\pi_1(s), \ldots, \pi_K(s))^\top$ for any location $s \in \mathcal{D}$, where $\sum_{k=1}^K \pi_k(s) = 1$ and $\pi_k(s) > 0$.

We propose a Multinomial-KSBP that models location-dependent multinomial parameters through an infinite mixture with spatial kernels:

\begin{enumerate}[1)]
\setlength{\parskip}{0cm}
\setlength{\itemsep}{0cm}
    \item Spatial location parameters: $\Gamma_h \sim H$ (uniform over $\mathcal{D}$)
    \item Stick-breaking parameters: $V_h \sim \text{Beta}(1, \lambda)$
    \item Component parameters: $\boldsymbol{\theta}_h \sim \text{Dirichlet}(\gamma_0/K \cdot \mathbf{1}_K)$
\end{enumerate}

The location-dependent probability vector is $\boldsymbol{\pi}(s) = \sum_{h=1}^{\infty} \pi_h(s) \boldsymbol{\theta}_h$, where the spatially-adaptive weights are:
$$
\pi_h(s) = V_h K(s, \Gamma_h) \prod_{l=1}^{h-1} [1 - V_l K(s, \Gamma_l)]
$$

Here, $K(s, \Gamma_h)$ is a Gaussian kernel controlling spatial dependence. We introduce auxiliary variables $z_i$ where $P(z_i = h \mid \mathbf{V}, \boldsymbol{\Gamma}) = \pi_h(s_i)$, giving: $z_i \mid \mathbf{V}, \boldsymbol{\Gamma}, s_i \sim \text{Categorical}(\pi_1(s_i), \pi_2(s_i), \ldots)$ and $\mathbf{y}_i \mid z_i, \{\boldsymbol{\theta}_h\} \sim \text{Multinomial}(N_i, \boldsymbol{\theta}_{z_i})$.

For posterior inference, we develop an efficient Gibbs sampling algorithm incorporating slice sampling techniques to handle the infinite mixture structure, with Metropolis-Hastings updates for spatial location parameters.

\vspace{-0.4cm}
\section{Empirical Results}
\vspace{-0.25cm}
As an illustrative application, we apply the model to archaeological obsidian data from Japan, successfully capturing spatial patterns with uncertainty estimates. Sparse data regions show elevated uncertainty, demonstrating principled handling of data scarcity. The results reveal temporal evolution in obsidian trade networks, providing evidence of changes in prehistoric exchange systems. Our approach offers a flexible nonparametric Bayesian framework for spatial multinomial analysis with rigorous uncertainty quantification. Future work includes semiparametric extensions and Gaussian Process connections.

\vspace{-0.1cm}
\footnotesize
{
\begin{thebibliography}{9}
\bibitem[1]{Dunson2008-dg} Dunson, D. B. and Park, J. H. (2008). {\it Kernel stick-breaking processes}. Biometrika, 95(2), 307--323.
\bibitem[2]{Walker2007-aj} Walker, S. G. (2007). {\it Sampling the Dirichlet mixture model with slices}. Communications in Statistics - Simulation and Computation, 36(1), 45--54.
\bibitem[3]{Polson2013} Polson, N. G., Scott, J. G., and Windle, J. (2013). {\it Bayesian inference for logistic models using Pólya–Gamma latent variables}. Journal of the American Statistical Association, 108(504), 1339--1349.
\end{thebibliography}
}

\end{document}
