% !TEX program = pdflatex
\documentclass[dvipdfmx,11pt]{beamer}

% Theme
\usetheme{Madrid}
\usecolortheme{seahorse}

% Packages
\usepackage{amsmath, amssymb, bm}
\usepackage{graphicx}
\usepackage{hyperref}
\usepackage{mathtools}
\usepackage{physics}
\usepackage{booktabs}
\usepackage{siunitx}
\usepackage{algorithm}
\usepackage{algpseudocode}

% Title
\title[Multinomial KSBP]{Kernel Stick-Breaking Process と\newline 空間多項データへの拡張}
\author[太田]{太田}
\institute[]{Bayesian Statistics Project}
\date{\today}

% Bibliography
\bibliographystyle{apalike}

\begin{document}

% Title page
\begin{frame}
  \titlepage
\end{frame}

% Agenda
\begin{frame}{アジェンダ}
  \tableofcontents
\end{frame}

\section{背景と目的}
\begin{frame}{背景}
  \begin{itemize}
    \item 黒曜石の\textbf{産地構成比}を地理位置に依存して推定したい
    \item 頻度論の\textbf{Nadaraya–Watson (NW)} はシンプルだが不確実性表現が限定的
    \item \textbf{Kernel Stick-Breaking Process (KSBP)} による\textbf{ノンパラメトリック・ベイズ回帰}を導入
  \end{itemize}
\end{frame}

\section{KSBPの定義}
\begin{frame}{設定}
  説明変数 $x \in \mathcal X$ に依存する確率測度族 $\{G_x\}$ を考える.
  \begin{equation*}
    f(y\mid x) = \int f(y\mid x, \phi)\, dG_x(\phi)
  \end{equation*}
  目的は $\{G_x\}$ に\textbf{柔軟な事前分布}を与えること(Dunson \\& Park, 2008).
\end{frame}

\begin{frame}{KSBPの構成}
  独立な無限列 $\{\Gamma_h, V_h, G_h^*\}_{h=1}^\infty$ を導入:
  \begin{itemize}
    \item $\Gamma_h \sim H$:位置(カーネル中心)
    \item $V_h \sim \mathrm{Beta}(a_h,b_h)$:スティック長
    \item $G_h^* \sim Q$:混合成分の基準測度
  \end{itemize}
  \vspace{0.5em}
  \begin{block}{定義(KSBP)}
    \begin{align*}
      G_x
      &= \sum_{h=1}^\infty \pi_h(x)\, G_h^*, \\
      \pi_h(x)
      &= W_h(x) \prod_{\ell<h}\bigl[1-W_\ell(x)\bigr], \\
      W_h(x)
      &= V_h\, K(x,\Gamma_h),\quad K: \mathcal X\times\mathcal L \to [0,1].
    \end{align*}
  \end{block}
  $K$ は距離減衰を与えるカーネル(例:RBF).
\end{frame}

\begin{frame}{特別な場合}
  \begin{itemize}
    \item $K(x,\Gamma)\equiv 1$ なら $G_x\equiv G$:Dirichlet過程のstick-breakingを再現
    \item $G_h^*=\delta_{\theta_h}$ で離散混合:標準的なDP混合に一致
    \item $\{a_h,b_h\}$ の設定でPitman–Yor等も包含
  \end{itemize}
\end{frame}

\section{Multinomial-KSBP}
\begin{frame}{空間多項データへの拡張}
  \begin{itemize}
    \item 各クラスタ $h$ に多項パラメータ $\bm{\theta}_h \sim \mathrm{Dir}(\gamma_0/K,\dots)$ を割当
    \item 位置 $s$ での産地比 $\bm{\pi}(s)=\sum_{h=1}^\infty \pi_h(s)\, \bm{\theta}_h$
    \item 観測:$\bm{y}_i\mid s_i \sim \mathrm{Multinomial}(N_i,\bm{\pi}(s_i))$
  \end{itemize}
\end{frame}

\begin{frame}{推論(概略)}
  \begin{itemize}
    \item $\bm{\theta}_h\mid \{z_i,\bm{y}_i\}$:\textbf{Dirichlet} による更新
    \item $z_i$(クラスタ割当):未規格化重み $w_{ih}=\pi_h(s_i)\prod_k \theta_{hk}^{y_{ik}}$ からカテゴリカル
    \item $V_h$:\textbf{slice sampling}(Walker, 2007)でBeta更新
    \item $\Gamma_h$:ランダムウォークMH(カーネル中心の更新)
  \end{itemize}
\end{frame}

\section{実装と設定}
\begin{frame}{実装場所(本リポジトリ)}
  \begin{itemize}
    \item 実装:\texttt{bayesian\_statistics/models/composition/ksbp.py}
    \item 設定:\texttt{bayesian\_statistics/models/config/model\_config.py}(\texttt{ksbp\_...} パラメータ)
    \item ノートブック:\texttt{notebooks/20\_model\_KSBP.ipynb}
  \end{itemize}
\end{frame}

\begin{frame}{ハイパーパラメータ例}
  \begin{itemize}
    \item カーネル幅(座標・コスト・標高・方位・河川)
    \item Stick尺度 $\lambda$,Dirichlet集中度 $\gamma_0$,トランケーション $J_{\max}$
    \item 反復 $N$,バーンイン,シードなど
  \end{itemize}
\end{frame}

\section{結果のイメージ}
\begin{frame}{結果(要約)}
  \begin{itemize}
    \item 位置依存の\textbf{後分布平均}と\textbf{標準偏差}により不確実性を可視化
    \item \textbf{データ稀薄領域での分散増大}がNWより適切に表現
    \item 図は \texttt{docs/0701\_presentation\_draft.tex} の該当スライド参照
  \end{itemize}
\end{frame}

\section{NWとの比較}
\begin{frame}{NW vs KSBP}
  \begin{itemize}
    \item \textbf{NW}: 平滑化の強さは帯域幅依存,不確実性の扱いが限定的
    \item \textbf{KSBP}: 混合による柔軟な形状,\textbf{完全なベイズ的不確実性}の定量化
    \item モデル選択や比較はLOOCV/WAIC等で検討(実装済)
  \end{itemize}
\end{frame}

\section{まとめ}
\begin{frame}{まとめと今後}
  \begin{itemize}
    \item KSBPを空間多項データへ拡張し,産地構成比の不確実性を評価
    \item 特徴量を併用したKSBP(座標以外のカーネル)で精度向上余地
    \item NNGP/GP系モデルとのベンチマーク,スライス+MHの効率化
  \end{itemize}
\end{frame}

\section*{参考文献}
\begin{frame}{参考文献}
\small
\begin{thebibliography}{9}
\bibitem[Dunson and Park, 2008]{Dunson2008-dg}
  Dunson, D. B. and Park, J. H. (2008).
  Kernel stick-breaking processes.
  \emph{Biometrika}, 95(2), 307--323.
\end{thebibliography}
\end{frame}

\end{document}
