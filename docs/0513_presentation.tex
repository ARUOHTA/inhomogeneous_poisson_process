\documentclass[xelatex, 8pt]{beamer}
\mode<presentation>{\usetheme{Dresden}}
% Boadilla
\usecolortheme[RGB={22, 74, 132}]{structure}
%\usefonttheme{professionalfonts}

\usepackage{xeCJK}
\setCJKmainfont{Noto Serif CJK JP}
%\renewcommand{\familydefault}{\sfdefault}

% IPAexMincho
% Noto Serif CJK JP

\usepackage{tikz}
\usetikzlibrary{intersections, calc, arrows.meta}

%seagullいいね
%
\usepackage{amsmath}
\usepackage{amsthm}
\usepackage{algorithm}
\usepackage{algorithmic}
\usepackage{subcaption}  %図を横に配置してそれぞれにキャプションを追加

% 定理環境で使う言葉を用意
\theoremstyle{plain}
\newtheorem{thm}{Theorem}
\newtheorem*{thm*}{Theorem}

\theoremstyle{definition}
\newtheorem{dfn}{Definition}

% フォントサイズの設定
\setbeamerfont{itemize/enumerate body}{size=\normalsize}
\setbeamerfont{itemize/enumerate subbody}{size=\normalsize}
\setbeamerfont{itemize/enumerate subsubbody}{size=\normalsize}

% リンクに関するセットアップ
\usepackage{url}

\usepackage[dvipdfmx]{color, hyperref}
\usepackage{cite}

% beamerではなぜかこれが必要らしい
%\hypersetup{pdfborder={0 0 1}}
\usepackage{xcolor}
\hypersetup{
	colorlinks=true,
	citecolor=blue,
	linkcolor=red,
	urlcolor=orange,
}

%ページ番号
\setbeamertemplate{footline}[frame number]

%%%%%%%%%%%%%%%%%%%%%%%%%%%%%% Metadata %%%%%%%%%%%%%%%%%%%%%%%%%%%%%%
\hypersetup
{
	%Separate multiple authors by comma
	pdfauthor={},
	pdftitle={Exact Bayesian Model for Non-Homogeneous Poisson Processes and Its Application to Obsidian Data},
	pdfsubject={},
	pdfkeywords={},
	colorlinks=false
}

%%%%%%%%%%%%%%%%%%%%%%%%%%%%%% Title related %%%%%%%%%%%%%%%%%%%%%%%%%%%%%%

\title[Contact: Aru Ohta (otaru1214@gmail.com)]{ノンパラメトリック法による黒曜石の産地構成比の推定}
\subtitle{Non-Parametric Models for Multinomial Point-Referrenced Data and Its Application to Obsidian Data}
\date[2025]{2025-5-13}
\author[M2 Aru Ohta]{M2 Aru Ohta}
\institute[Kyoto University]{京都大学情報学研究科}


%%%%%%%%%%%%%%%%%%%%%%%%%%%%% Presentation begins here %%%%%%%%%%%%%%%%%%%%%%%%%



\begin{document}

\frame{\titlepage}

\begin{frame}
{\Large 目次 Contents}
 \tableofcontents
\end{frame}

\section{背景と目的}

\begin{frame}
{\Large 目次}
 \tableofcontents[currentsection]
\end{frame}

\begin{frame}{黒曜石の流通と交易システム}
    \begin{itemize}
        \item 現生人類が日本列島に移動してきたのは約38,000年前だが、中国から漢字が伝わり、初めて文字が使用されたのは約2,000年前
        \item その間の人々の移動・生活を実証的な方法で復元するほぼ唯一の方法: 遺跡から発見される遺構や遺物
        \item 特に旧石器時代から縄文時代にかけて、\textbf{黒曜石(Obsedian)}は石器を作るための優れた材料として使用されてきた
        \item 縄文時代になると、黒曜石は限られた原産地の集落だけでなく、異なる集団間に広く流通するようになり、\textbf{交易システムが出現}していたことがわかっている
        \item 流通パターンと時空間的分布の定量的解明が重要
    \end{itemize}

    \begin{figure}
        \centering
        \includegraphics[width=0.3\linewidth]{fig/obsedian.png}
        \caption{黒曜石原石(前18,000年)。\href{https://bunka.nii.ac.jp/heritages/detail/515934}{文化遺産オンライン}より引用。}
        \label{fig:enter-label}
    \end{figure}

\end{frame}

\begin{frame}{研究目的}

    \begin{itemize}
        \item 目的:任意の地点における黒曜石の\textbf{産地構成比}を推定する
        \\[2mm]
        \item そのために、
        \\[2mm]
            \begin{itemize}
                \item 黒曜石が出土した遺跡の位置
                \item それぞれの遺跡から出土したそれぞれの産地の黒曜石の出土数
            \end{itemize}
            \\[2mm]
            の2つをモデル化することを考える。
            \\[2mm]
            前者についてのモデルを\textbf{遺跡の存在確率モデル}、後者についてのモデルを\textbf{産地構成比モデル}とよぶことにする。

    \end{itemize}
\end{frame}

\section{データの概要}
\begin{frame}
{\Large 目次}
 \tableofcontents[currentsection]
\end{frame}


\begin{frame}{データの基本情報}
    \begin{itemize}
        \item \textbf{対象領域}: 関東地方(北緯34-37度、東経138-141度)
        \item \textbf{データ規模}:
        \begin{itemize}
            \item 遺跡数: 224箇所
            \item 総出土数: 31,244点
        \end{itemize}
        \vspace{2mm}
        \item \textbf{時期区分}:
        \begin{itemize}
            \item 早期・早々期 (約12,000年前〜7,000年前)
            \item 前期 (約7,000年前〜5,500年前)
            \item 中期 (約5,500年前〜4,500年前)
            \item 後期 (約4,500年前〜3,500年前)
            \item 晩期 (約3,500年前〜2,800年前)
        \end{itemize}
        \vspace{2mm}
        \item \textbf{産地分類}:
        \begin{itemize}
            \item 神津島
            \item 信州(和田峠、男女倉、諏訪、蓼科)
            \item 箱根
            \item 高原山
        \end{itemize}
    \end{itemize}
\end{frame}

\begin{frame}{データの概要}
    \begin{figure}
        \centering
        \includegraphics[width=0.85\linewidth]{fig/obsedian_variables.png}
        \label{fig:enter-label}
    \end{figure}
\end{frame}


\section{遺跡の存在確率モデル}
\begin{frame}
{\Large 目次}
 \tableofcontents[currentsection]
\end{frame}


\begin{frame}{点過程とは}

\textbf{点過程 (Point Process)}: 連続的な時間や空間上に発生する離散的なイベントの発生位置、発生時刻を記述する確率過程

歴史的には地震のモデル,株の板情報のモデル,感染症のモデル,脳のニューロンのモデルなどに使われてきた

\begin{figure}
    \centering
    \begin{subfigure}{0.4\textwidth}
        \centering
        \includegraphics[width=\textwidth]{fig/pointprocess.png}
        \caption{時間点過程の例}
        \label{fig:pointprocess1}
    \end{subfigure}
    \hfill
    \begin{subfigure}{0.4\textwidth}
        \centering
        \includegraphics[width=\textwidth]{fig/pointprocess2.png}
        \caption{空間点過程の例}
        \label{fig:pointprocess2}
    \end{subfigure}
    \label{fig:both-predictions}
\end{figure}

\end{frame}

\begin{frame}{非斉次ポアソン過程の定義}
    \begin{dfn}[ポアソン過程]
        $X$を$\mathcal{D}$上の計数過程とし、$\lambda(s)$を$\mathcal{D}$上の連続な非負実数値可測関数とする。

        任意の(可測)集合$D \subset \mathcal{D}$に対して、
        $$
        \Lambda(D) := \int_D \lambda(s) ds
        $$
        とする。ここで、
        $$
        X(D) \sim \text{Poisson}(\Lambda(D))
        $$
        であり、かつ、任意の$k\in \mathbb{N}$に対して、互いに素な集合$D_1, D_2, \ldots , D_k$に対して、$X(D_1), X(D_2), \ldots , X(D_k)$が互いに独立であるとき、$X$を\textbf{強度$\lambda(s)$を持つ非斉次ポアソン過程}とよび、
        $$
        X \sim \text{IPP}(\lambda)
        $$
        とかく。
    \end{dfn}
\end{frame}

\begin{frame}{非斉次ポアソン過程の尤度関数}

    非斉次ポアソン過程$X$に対する統計的推測を行うために、データ生成過程を考える。
    \begin{itemize}
        \item 観測データは領域$\mathcal{D}$全体のイベント発生数$n_X$と、それぞれのイベント発生位置$s_i (i = 1, \ldots , n_X)$の組であり、それらの分布は次のように定義される
        $$
        X \sim \text{IPP}(\lambda) \quad \Leftrightarrow \quad n_X \sim \operatorname{Poisson}\left(\Lambda(\mathcal{D})\right), \quad s_i \stackrel{\mathrm{iid}}{\sim} \frac{\lambda\left(s\right)}{\Lambda(\mathcal{D})}, \quad i=1, \ldots, n_X
        $$
        \item したがって、非斉次ポアソン過程の尤度関数は以下のようにかける
            $$
            \begin{aligned}
            P(X \mid \lambda)
            &= \frac{\left[\Lambda(\mathcal{D})\right]^{n_X} e^{-\Lambda(\mathcal{D})}}{n_X!} \cdot \prod_{i=1}^{n_X} \frac{\lambda(s_i)}{\Lambda(\mathcal{D})}\\
            &= \frac{e^{-\Lambda(\mathcal{D})}}{n_X!} \cdot \prod_{i=1}^{n_X} \lambda(s_i)\\
            &=  \exp\left( -\int_{\mathcal{D}} \lambda(s) \, ds \right) \cdot \frac{1}{n_X!}\cdot \prod_{i=1}^{n_X} \lambda(s_i)
            \end{aligned}
            $$
    \end{itemize}

    \textbf{ベイズ推論における課題}:
    \begin{itemize}
        \item 積分項$\int_{\mathcal{D}} \lambda(s) \, ds$の計算が困難
        \item 従来は離散近似が用いられてきたが、MCMCの効率が低下
    \end{itemize}
\end{frame}

\begin{frame}{潜在変数アプローチ(Moreira and Gamerman, 2022\cite{Moreira2022})}
    観測データ$X$はIPPに従うとする:
    $$
        X \sim \text{IPP}(\lambda)
    $$

    強度関数を以下のように分解:
    $$
    \lambda(s) = \lambda^* \cdot q(s), \quad s \in \mathcal{D}
    $$
    ここで:
    \begin{itemize}
        \item $\lambda^* > 0$: 強度関数の上限
        \item $q(s)$: 相対的な強度を表す関数(0から1の値となり、データの存在確率として解釈)
        $$
        q(s) = \frac{\exp(\boldsymbol{W}(s)^\top \boldsymbol{\beta})}{1 + \exp(\boldsymbol{W}(s)^\top \boldsymbol{\beta})}
        $$
    \end{itemize}

    新たな点過程$U$を潜在変数として導入:
    $$
    U \sim \text{IPP}(\lambda^*(1-q))
    $$
\end{frame}

\begin{frame}{潜在変数アプローチによる推定方法}
    尤度関数に含まれる計算不可能な積分項:
    $$
    P(X \mid \boldsymbol{\beta}, \lambda^*) = \exp\left( -\lambda^* \textcolor{red}{\int_{\mathcal{D}} q(s) \, ds} \right) \cdot \frac{(\lambda^*)^{n_X}}{n_X!} \prod_{i=1}^{n_X} q(s_i)
    $$
    ここで、$q(s) = \frac{\exp(\boldsymbol{W}(s)^\top \boldsymbol{\beta})}{1 + \exp(\boldsymbol{W}(s)^\top \boldsymbol{\beta})}$, $\boldsymbol{W}(s)$は位置$s$での説明変数

    \vspace{2mm}
    解決策: 潜在変数による拡張
    \begin{enumerate}
        \item \textbf{偽不在の点過程$U$を導入}:
        $$
        U \sim \text{IPP}(\lambda^*(1-q))
        $$

        \item \textbf{同時尤度を導出}:
        \begin{align*}
        P&(X, U \mid \boldsymbol{\beta}, \lambda^*) \\
        &= \exp \left( - \lambda^* |\mathcal{D}| \right) \cdot \frac{(\lambda^*)^{n}}{n_X!n_U!} \prod_{i=1}^{n} \frac{\left\{\exp\left( \boldsymbol{W}(s_i)^\top \boldsymbol{\beta} \right)\right\}^{y_i}}{1 + \exp\left(\boldsymbol{W}(s_i)^\top \boldsymbol{\beta}\right)}
        \end{align*}
        ただし、$n = n_X + n_U$, $y_i$は在/偽不在を表す二値変数
    \end{enumerate}
\end{frame}

\begin{frame}{パラメータと潜在変数の事後分布}
    \textbf{モデルの構成要素}:
    \begin{itemize}
        \item パラメータ: $\Theta = \{\lambda^*, \boldsymbol{\beta}\}$
            \begin{itemize}
                \item $\lambda^*$: 強度関数の上限値
                \item $\boldsymbol{\beta}$: 空間共変量の係数ベクトル
            \end{itemize}
        \item 潜在変数: $\{U, \boldsymbol{\omega}\}$
            \begin{itemize}
                \item $U$: 偽不在を表す点過程
                \item $\boldsymbol{\omega}$: Polya-Gamma変数\cite{Polson2013}($n$次元ベクトル)
            \end{itemize}
    \end{itemize}

    \vspace{3mm}
    \textbf{事後分布}:
    \begin{align*}
        \boldsymbol{\beta} \mid \boldsymbol{\omega}, X, U, \lambda^* &\sim \mathcal{N} \left( \boldsymbol{m}, V \right)\\
        &\text{ただし、} V = \left( B_0^{-1} + W^\top \Omega W \right)^{-1}\\
        &\boldsymbol{m} = V \left( B_0^{-1} \boldsymbol{b}_0 + W^\top \Omega \boldsymbol{z} \right)\\[2mm]
        \lambda^* \mid \boldsymbol{\beta}, \boldsymbol{\omega}, X, U &\sim \text{Ga}\left( m_0 + n, \quad r_0 + |\mathcal{D}| \right)\\[2mm]
        U \mid \boldsymbol{\beta}, \lambda^* &\sim \text{IPP}(\lambda^*(1-q))\\[2mm]
        \omega_i \mid \boldsymbol{\beta}, X, U &\sim \text{PG}(1, \boldsymbol{W}(s_i)^\top \boldsymbol{\beta})
    \end{align*}

    ここで、$W$は説明変数行列、$\Omega = \text{diag}(\omega_1, \ldots, \omega_n)$。
\end{frame}

\begin{frame}{ギブスサンプリングアルゴリズム}
    \begin{algorithm}[H]
    \caption{Gibbs Sampling Algorithm}
    \begin{algorithmic}
    \STATE Initialize $\lambda^{*(0)},\ \boldsymbol{\beta}^{(0)},\ \boldsymbol{\omega}^{(0)},\ U^{(0)}$
    \FOR{$\tau = 1$ to $T$}
        % Sample U
        \STATE \textbf{Step 1:} Sample $U^{(\tau)}$ using Poisson thinning
        \STATE \textbf{Step 2:} Update design matrix $W$
        % Sample beta
        \STATE \textbf{Step 3:} Sample $\boldsymbol{\beta}^{(\tau)} \sim \mathcal{N}(\boldsymbol{m}, V)$
        % Sample lambda
        \STATE \textbf{Step 4:} Sample $\lambda^{*(\tau)} \sim \text{Ga}(m_0 + n, r_0 + |\mathcal{D}|)$
        % Sample omega
        \STATE \textbf{Step 5:} Sample $\omega_i^{(\tau)} \sim \text{PG}(1, \tilde{\boldsymbol{W}}(s_i)^\top \boldsymbol{\beta}^{(\tau)})$
    \ENDFOR
    \end{algorithmic}
    \end{algorithm}

    \vspace{2mm}
    \textbf{Poisson thinningによる$U$のサンプリング}:
    \begin{enumerate}
        \item $N \sim \text{Poisson}(\lambda^* |\mathcal{D}|)$個の点を一様生成
        \item 各点$s_j$について確率$1-q(s_j)$で受理
        \item 受理された点の集合が$U^{(\tau)}$
    \end{enumerate}
\end{frame}

\begin{frame}\frametitle{実データでの推定結果}
\begin{figure}\centering\includegraphics[width=0.6\textwidth]{fig/trace_site_probability.png}\caption{各パラメータの事後分布とトレースプロット。上から、標高、傾斜角度、神津島からの距離、信州からの距離、箱根からの距離、高原山からの距離、河川からの距離}
\end{figure}
\end{frame}

\begin{frame}\frametitle{実データでの推定結果}
\begin{figure}\centering\includegraphics[width=0.8\textwidth]{fig/site_probability.png}\caption{領域全体での予測}
\end{figure}
\end{frame}

\begin{frame}\frametitle{ここまでのまとめ}
\begin{itemize}
    \item 遺跡の存在確率モデル:遺跡の位置を、非斉次ポアソン過程によってモデル化することができた。
    \item 次に、遺跡ごとの黒曜石の出土数とその比率のモデル化を行う。
    \vspace{3mm}
    \item 次節では、主に2つのモデルを扱う。
    \end{itemize}
\end{frame}

\section{産地構成比モデル}

\begin{frame}
{\Large 目次}
 \tableofcontents[currentsection]
\end{frame}

\begin{frame}{問題の定式化}

以下では、黒曜石の出土時期$t$についてはすべて独立に扱うことにする。

\begin{itemize}
\item 調査領域を$\mathcal{D} \subset \mathbb{R}^2$とする。遺跡$i = 1, ...., n_{X}$の位置を$s_i \in \mathcal{D}$とし、その地点における共変量を$\tilde{\boldsymbol{W}}(s_i)$とする。
\item そして、その遺跡における産地$k=1, \ldots, K$の黒曜石の出土数を$y_{ik}$とする。

\vspace{2mm}

ここで、調査領域内の任意の地点$s \in \mathcal{D}$における黒曜石の産地構成比

$$
\boldsymbol{\pi}(s) = (\pi_1(s), \ldots , \pi_{K}(s)), \quad \sum_{k=1}^{K}\pi_k(s) = 1$$

を求める問題を考える。
\end{itemize}
\end{frame}

\begin{frame}{モデル化の方針}

この章では、産地構成比を推定するために以下の2つのモデルを考える:

\begin{itemize}
    \item 頻度主義におけるノンパラメトリック法:Nadaraya-Watson推定量
    \item ベイズ統計におけるノンパラメトリック法:Kernel Stick-Breaking Process
\end{itemize}

\end{frame}

\subsection{モデル1: Nadaraya-Watson推定量}

\begin{frame}{ノンパラメトリック法による推定}

まず、以下のような単純な回帰モデルを考える。

$$
\frac{y_{ik}}{\sum_{k'}y_{ik'}} = \pi_k\left(s_i, \tilde{\boldsymbol{W}}(s_i)\right) + \epsilon, \quad k=1, \ldots, K
$$

そして、この回帰モデルにおける$\pi_k(\cdot)$を、以下のNadaraya-Watson推定量\cite{Kurisu2020-nh}によって求める:

$$
\hat{p}_k(s, \boldsymbol{w})=\frac{\sum_{i=1}^{n_X} K_h\left(s-s_i\right) \prod_{l=1}^{p}K_h(w_l - \tilde{\boldsymbol{W}_l}(s_i)) \cdot y_{i k}}{\sum_{i=1}^{n_X} K_h\left(s-s_i\right) \prod_{l=1}^{p}K_h(w_l - \tilde{\boldsymbol{W}_l}(s_i))\cdot \sum_{k^{\prime}} y_{i k^{\prime}}}
$$

ここで、$K_h$はカーネル関数で、以下のものを使用した:

$$
K_h\left(s-s_i\right)=\frac{1}{h^2} \exp \left(-\frac{d\left(s-s_i\right)^2}{2 h^2}\right)
$$

ここで、$h$はバンド幅であり、$d(\cdot)$は次に示す、Tober's Hiking Functionに基づくコスト関数である。

\end{frame}

\begin{frame}{Tobler's Hiking Function}

Tobler’s hiking function\cite{Tobler1999-gx}では、隣接する2地点間の移動速度(km/h)は以下のように定義される:

$$W = 6e^{-3.5\left|\frac{dh}{dx}+0.05\right|}$$

$$
\frac{dh}{dx} = S = \tan \theta
$$

ここで、$W$は歩行速度、$dh$は標高差、$dx$は距離、$S$は勾配、$\theta$は傾斜角を表す。

以下、この関数で求めた移動速度から移動コスト(分)を計算し、その経路上の最短距離を求めるアルゴリズムを実装する。

\end{frame}

\begin{frame}{Tobler's Hiking Function}

Tobler's Hiking Functionを調査領域全体に適用するために、移動のためのコストを設定する。

\begin{itemize}
    \item 陸上の2点間の移動コストは、上記のhiking functionでそのまま計算することができる。
    \item 海上の移動コストは、一般的な木造丸木船の移動速度を考慮して一律4km/hと設定する。
    \item 陸と海を行き来する移動コストは、沿岸部の地形によって異なるコストを設定する。
        \begin{itemize}
            \item 砂浜海岸だと思われる海岸線の移動コストは、通常の50倍、岩石海岸だと思われる海岸線の移動コストを無限大に設定する
        \end{itemize}
\end{itemize}
\begin{figure}\centering\includegraphics[width=0.4\textwidth]{fig/coast.png}\caption{沿岸部の地形例}
\end{figure}

\end{frame}

\begin{frame}{Tobler's Hiking Function}

領域全体で定義された移動コストを元に、以下の手順で領域内の任意の2点$s$, $s'$のコスト$d(s, s')$を定義し、カーネル関数として利用する。
\begin{itemize}
    \item まず、領域$\mathcal{D}$全体を、5次メッシュコード(250m単位)に従ってグリッド上に分割する。

\item 次に、各グリッドの中心点を計算し、グラフの頂点とする。

\item そして、東西南北の4方向に隣接するセル同士を辺で結び、その辺のコストとして、上のTobler's hiking functionから計算した移動時間(分)を設定する。

\item その上で、以下の再帰的な更新式に従って、すべての頂点$u$について、他のすべての頂点$v$までの最短コスト$C_u(v)$ を求める:

$$
C_u(v) \leftarrow \min\left(\{C_u(v)\}\cup \{C_u(w) + t_{w\rightarrow v} \mid w \in \mathcal{N}_v\}\right)
$$

\begin{itemize}
    \item $t_{w\rightarrow v}$ :頂点 $w$ から隣接セル $v$ への移動コスト
    \item $\mathcal{N}_v$:頂点$v$の隣接頂点の集合
\end{itemize}

\item 領域内の任意の2点$s$, $s'$について、2点間の距離を、それぞれが属しているグリッドの中心点$v_s$, $v_{s'}$のコストの行き帰りの平均として定義する:
$$d(s, s') := \frac{C_{v_s}(v_{s'}) + C_{v_s}(v_{s'})}{2}$$

\end{itemize}

\end{frame}

\begin{frame}{Tobler's Hiking Function}
\begin{figure}\centering\includegraphics[width=0.8\textwidth]{fig/tobler.png}\caption{Tobler's Hiking functionに基づく移動コストの例。ヒートマップは赤色の点からの移動コストを示す。}
\end{figure}
\end{frame}

\begin{frame}{Nadaraya-Watson推定量の推定結果}

$$
\hat{p}_k(s, \boldsymbol{w})=\frac{\sum_{i=1}^{n_X} K_h\left(s-s_i\right) \prod_{l=1}^{p}K_h(w_l - \tilde{\boldsymbol{W}_l}(s_i)) \cdot y_{i k}}{\sum_{i=1}^{n_X} K_h\left(s-s_i\right) \prod_{l=1}^{p}K_h(w_l - \tilde{\boldsymbol{W}_l}(s_i))\cdot \sum_{k^{\prime}} y_{i k^{\prime}}}
$$

$$
K_h\left(s-s_i\right)=\frac{1}{h^2} \exp \left(-\frac{d\left(s-s_i\right)^2}{2 h^2}\right)
$$

\begin{itemize}
    \item \textbf{対象領域}: 関東地方(北緯34-37度、東経138-141度)
    \item \textbf{データ規模}:
    \begin{itemize}
        \item 遺跡数: 224箇所
        \item 総出土数: 31,244点
    \end{itemize}
    \vspace{2mm}
    \item \textbf{時期区分}:
    \begin{itemize}
        \item 早期・早々期 (約12,000年前〜7,000年前)
        \item 前期 (約7,000年前〜5,500年前)
        \item 中期 (約5,500年前〜4,500年前)
        \item 後期 (約4,500年前〜3,500年前)
        \item 晩期 (約3,500年前〜2,800年前)
    \end{itemize}
    \vspace{2mm}
    \item \textbf{産地分類}:
    \begin{itemize}
        \item 神津島
        \item 信州(和田峠、男女倉、諏訪、蓼科)
        \item 箱根
        \item 高原山
    \end{itemize}
\end{itemize}

\end{frame}

\begin{frame}{Nadaraya-Watson推定量の推定結果}

\begin{figure}\centering\includegraphics[width=0.9\textwidth]{fig/obsidian_ratio_all_NW.png}\caption{横軸が時期(左から順)、縦軸が産地(上から神津島、信州、箱根、高原山)を表す}
\end{figure}

\end{frame}

\begin{frame}{Nadaraya-Watson推定量の推定結果}
\begin{figure}\centering\includegraphics[width=1\textwidth]{fig/obsidian_ratio_0_神津島.png}\caption{早期・早々期、神津島}
\end{figure}
\end{frame}

\begin{frame}{ここまでのまとめとここからの概略}

ここまで、遺跡の存在確率のモデル(非斉次ポアソン過程)と、黒曜石の産地構成比の頻度論モデル(Nadaraya-Watson推定量)を実装した。

\vspace{3mm}

これらのモデルを拡張あるいは統一的な扱いが求められる:

\begin{itemize}
    \item 遺跡の存在確率に依存した形の産地構成比のモデル化
    \item 産地構成比に関するさらなる詳細なモデル化(例:線形効果と非線形効果の分離)
\end{itemize}

また、統計モデルの新規性という観点でも、拡張したモデルの探索を続けたい。

\vspace{3mm}
→そのために、まずは後者の産地構成比のモデルについて、ベイズモデルへと拡張するという方針を考える。

本セクションでは、まずKernel Stick-Breaking Processというモデルについて説明し、その後、その応用として新たな産地構成比のベイズモデルを提案する。


\end{frame}

\subsection{Kernel Stick-Breaking Process}

\begin{frame}{Kernel Stick-Breaking Process}
Dunson and Park (2008)\cite{Dunson2008-dg}は、回帰モデルのための柔軟なノンパラメトリックベイズモデルの一つであるKernel Stick-Breaking Processを提案した。

ノンパラメトリックベイズにおける目的は、予測変数 $x$ に依存する確率分布族 $\{G_x : x \in \mathcal{X}\}$ に対して、柔軟な事前分布を定義することである。

\vspace{1em}
\textbf{設定:}
\begin{itemize}
    \item $\mathcal{X}$ はユークリッド空間上のルベーグ可測集合
    \item 各 $G_x$ は確率空間 $(\Omega, \mathcal{F})$ 上の確率測度
\end{itemize}

\vspace{1em}
\textbf{目的:以下の回帰モデル(予測分布)を推定するために、$G_x$の事前分布を構成すること}
\[
f(y \mid x) = \int f(y \mid x, \phi) \, dG_x(\phi)
\]
\begin{itemize}
    \item $f(y \mid x, \phi)$:既知の観測モデル
    \item $G_x$:未知の混合測度
\end{itemize}

\end{frame}

\begin{frame}{Kernel Stick-Breaking Process (KSBP) の定義}
説明変数$x \in \mathcal{X}$に依存する確率測度の族 $\{G_x\}$ に対して、カーネル付きStick-Breaking表現を用いた事前分布(Kernel Stick-Breaking Process: KSBP)を導入する。

\vspace{1em}
\textbf{定義対象:}
\begin{itemize}
    \item $\mathcal{X} \subset \mathbb{R}^p$: ユークリッド空間上の集合
    \item $(\Omega, \mathcal{F})$: 標本空間とボレル$\sigma$-加法族
    \item $\{G_x\}_{x \in \mathcal{X}}$: 各$x$に対応する確率測度
\end{itemize}

\end{frame}

\begin{frame}{Kernel Stick-Breaking Process (KSBP) の定義}
まず、以下の独立なランダム要素の無限列を導入する:

\[
\{ \Gamma_h, V_h, G_h^* \}_{h = 1}^{\infty}
\]

\begin{itemize}
    \item $\Gamma_h \overset{\text{iid}}{\sim} H$:位置パラメータ(場所)
    \item $V_h \overset{\text{ind}}{\sim} \mathrm{Beta}(a_h, b_h)$:重み生成用の確率変数
    \item $G_h^* \overset{\text{iid}}{\sim} Q$:確率測度(ランダムなベース分布)
\end{itemize}

\vspace{1em}
\textbf{空間と測度:}
\begin{itemize}
    \item $H$は$(\mathcal{L}, \mathcal{A})$上の確率測度。$\mathcal{L} \subseteq \mathbb{R}^p$
    \item $Q$は$(\Omega, \mathcal{F})$上の確率測度に対する確率測度(つまり、確率測度上の測度)
\end{itemize}

\vspace{0.5em}
この無限列を用いて、KSBPを構成していく。
\end{frame}

\begin{frame}{Kernel Stick-Breaking Process (KSBP) の定義}
上で導入した成分を用いて、$G_x$は以下のように構成される:

\vspace{0.5em}
\begin{block}{Definition: KSBP}
\[
G_x = \sum_{h=1}^\infty \pi_h(x; V_h, \Gamma_h) G_h^*
\]
ただし、
\[
\pi_h(x; V_h, \Gamma_h) = W(x; V_h, \Gamma_h) \prod_{l < h} \left[1 - W(x; V_l, \Gamma_l)\right]
\]
\[
W(x; V_h, \Gamma_h) = V_h \cdot K(x, \Gamma_h)
\]
\end{block}

\vspace{1em}
\textbf{カーネル関数} $K : \mathcal{X} \times \mathcal{L} \rightarrow [0, 1]$ は、距離に応じて重みを調整するもので、たとえば $K(x, \Gamma) = \exp(-\psi \|x - \Gamma\|)$ などが使われる。

直観的には、
\begin{itemize}
    \item 各項 $\pi_h(x)$ は「残った棒」のうち $h$番目に対応する割合
    \item $K(x, \Gamma_h)$ によって、$x$から遠い$\Gamma_h$には小さな重みしか与えられない
    \item よって、近傍の成分に集中し、$x$に依存した局所的な混合が実現される
\end{itemize}

\end{frame}


\begin{frame}{Dirichlet過程}
\textbf{Dirichlet過程 (DP)} は、確率測度 $G$ に対する事前分布として広く用いられている確率過程である。

\begin{block}{定義(Ferguson, 1973)}
$G \sim \mathrm{DP}(\alpha G_0)$ は、任意の有限分割 $B_1, \dots, B_k$ に対し:
\[
(G(B_1), \dots, G(B_k)) \sim \mathrm{Dir}(\alpha G_0(B_1), \dots, \alpha G_0(B_k))
\]
\end{block}

\vspace{1em}
\textbf{Stick-breaking表現(Sethuraman, 1994)}により、$G$は次のように構成可能:
\[
G = \sum_{h=1}^\infty p_h \delta_{\theta_h}, \quad
p_h = V_h \prod_{l=1}^{h-1}(1 - V_l)
\]
\begin{itemize}
    \item $V_h \sim \mathrm{Beta}(1, \alpha)$
    \item $\theta_h \sim G_0$
\end{itemize}

この構成により、$G$ は離散的な測度となる。
\end{frame}



\begin{frame}{KSBPの特殊ケース}
KSBPは、既存の多くのノンパラメトリックベイズモデルを含む一般化されたモデルである。

\vspace{1em}
\textbf{例1:カーネル $K(x, \Gamma) \equiv 1$ の場合}
\begin{itemize}
    \item $\Rightarrow$ $G_x = G$:すべての$x$に対して共通な混合分布
    \item $G$ は以下のような構成を持つ:
    \[
    G = \sum_{h=1}^\infty \left( V_h \prod_{l < h} (1 - V_l) \right) G_h^*
    \]
\end{itemize}

\vspace{1em}
\textbf{例2:$G_h^* = \delta_{\theta_h}$ とした場合}
\begin{itemize}
    \item $\Rightarrow$ Ishwaran & James (2001) の stick-breaking mixture (Dirichlet Process Gaussian Mixture Modelの一種)
\end{itemize}

\vspace{1em}
他にも、
\begin{itemize}
    \item $a_h = 1 - a$, $b_h = b + h a$:Pitman-Yor過程
    \item $G_h^* \sim \mathrm{DP}(\alpha G_0)$:DP混合の混合(DPの2階拡張)
\end{itemize}
\end{frame}


\begin{frame}{KSBPの特殊ケース:Dirichlet Processとの関係}
KSBPの一般形式を復習する:

\[
G_x = \sum_{h=1}^\infty \pi_h(x) G_h^*, \quad
\pi_h(x) = W(x; V_h, \Gamma_h) \prod_{l < h} [1 - W(x; V_l, \Gamma_l)], \quad
W(x; V_h, \Gamma_h) = V_h K(x, \Gamma_h)
\]

\vspace{1em}
\textbf{例1:} $K(x, \Gamma) \equiv 1$ のとき:
\begin{itemize}
    \item $G_x = G$: 全ての$x$に対して同一の混合測度
    \item DPの stick-breaking 表現を再現:
    \[
    G = \sum_{h=1}^\infty p_h G_h^*, \quad p_h = V_h \prod_{l < h} (1 - V_l)
    \]
\end{itemize}

\vspace{1em}
\textbf{例2:} $G_h^* = \delta_{\theta_h}$ ならば:
\begin{itemize}
    \item 離散的な混合測度となり、Dirichlet Process Mixtureモデルとなる
\end{itemize}
\end{frame}


\begin{frame}{モデル2: Multinomial-KSBP}

このKernel Stick-Breaking Processを空間データに応用して、産地構成比を推定するためのノンパラメトリックベイズモデルを考えたい。

$K$個の非負整数の応答変数を持つため、これをMultinomial Kernel Stick-Breaking Processと呼ぶことにする。

\end{frame}


\subsection{モデル2: Multinomial-KSBP}

\begin{frame}{Multinomial-KSBP}
観測空間 $\mathcal X\subset\mathbb R^{d}$ とし,KSBPと同様に、独立な無限列
\[
\{\Gamma_h,\;V_h,\;G_h^\* \}_{h=1}^{\infty}
\]
を導入する.

\vspace{0.5em}
それぞれの事前分布を以下のようにする:
\[
\Gamma_h\sim H,\qquad
V_h\sim\operatorname{Beta}(a_h,b_h),\qquad
G_h^* \sim Q
\]

\vspace{0.5em}
ここで:
\begin{itemize}
  \item $\Gamma_h$ :空間的位置。$H$は観測空間内の一様分布とする
  \item $V_h$ :stick-breaking の比率
  \item $G_h^*$ :混合成分
\end{itemize}
\end{frame}

\begin{frame}{Multinomial-KSBP}
空間的局所性を反映するため,カーネル関数 $K:\mathcal X\times\mathcal X\to[0,1]$ を導入し,以下のように重みを定義:

\[
W_h(x) = V_h\,K(x,\Gamma_h)
\]
ここでも、ガウスカーネルを利用する:
\[
K(x,\Gamma) = \exp\!\left\{-\frac{\|x - \Gamma\|^2}{2h^2}\right\}
\]

\vspace{1em}
これを用いて,$x$ における stick-breaking 重みを以下で構成する:

\[
\pi_h(x) = W_h(x)\prod_{l < h}[1 - W_l(x)], \quad h=1,2,\dots
\]

\vspace{0.5em}
このとき,$\sum_h \pi_h(x) = 1$ はほとんどいたるところでで成り立つ。
\end{frame}

\begin{frame}{Multinomial-KSBP}
KSBPと同様に、$x$ に依存する確率測度 $G_x$ を以下で定義する:

\[
G_x = \sum_{h=1}^{\infty} \pi_h(x)\,G_h^*
\]

\begin{itemize}
  \item $\pi_h(x)$:空間位置 $x$ に依存した重み
  \item $G_h^*$:クラスタ $h$ に対応する測度
\end{itemize}

\vspace{1em}
この構成により,$G_x$ は場所ごとに異なる分布を持つランダム測度となる。
\end{frame}

\begin{frame}{Multinomial-KSBP}
クラスタhにおける混合成分は、ディリクレ分布から生成される確率ベクトルの値を確率1でとるものとする。直感的には、$\theta_h$ は、$\Gamma_h$を中心とするクラスタhにおける産地構成比を表す。

\[
\theta_h \mid \gamma_0 \sim
\operatorname{Dirichlet}\left(\tfrac{\gamma_0}{K},\dots,\tfrac{\gamma_0}{K}\right),
\quad
G_h^* = \delta_{\theta_h}
\]

\vspace{0.5em}
ここで、$\delta_{\theta_h}$はディラック測度。これにより,$G_x$ はカテゴリカル分布の加重平均として構成される。

観測データ $\mathbf{y}_i \in \mathbb{N}^K$ は多項分布から生成されるとする:

\[
\pi(s) = \sum_{h=1}^{\infty} \pi_h(s)\,\theta_h
\]
\[
\mathbf{y}_i \mid \pi(s_i) \sim \operatorname{Multinomial}(N_i,\; \pi(s_i))
\]

\end{frame}

\begin{frame}{モデルの階層構造}
\[
\begin{aligned}
\Gamma_h &\sim H, \\
\quad V_h &\sim \operatorname{Beta}(1, \lambda), \\
\theta_h &\sim \operatorname{Dirichlet}\left(\tfrac{\gamma_0}{K}\mathbf{1}_K\right), \\
\pi_h(s) &= V_h K_h(s) \prod_{l < h} [1 - V_l K_l(s)], \\
\pi(s) &= \sum_h \pi_h(s)\, \theta_h, \\
\mathbf{y}_i \mid \pi(s_i) &\sim \operatorname{Multinomial}(N_i,\; \pi(s_i))
\end{aligned}
\]

\end{frame}

%----------------------------------------------------------------------
\begin{frame}{補助変数 \(\boldsymbol z=(z_1,\dots,z_n)\) の導入}
\begin{itemize}
%-------------------------------------------------
\item
      無限混合 \(\displaystyle\pi(s)=\sum_{h\ge1}\pi_h(s)\theta_h\) を
      サンプリング可能な形に分解するために、補助変数として$z_i$を導入し、
%-------------------------------------------------
\item
      各観測点 \(s_i\) について離散変数
      \[
        z_i \;\in\;\{1,2,\dots\},\qquad
        \mathbb P(z_i=h\mid\mathbf V,\Gamma)
        \;=\;\pi_h(s_i)
      \]
      と定義する。
%-------------------------------------------------
\item これによって、データ生成過程は
      \[
        %
        z_i\mid \mathbf V,\Gamma,s_i \sim \mathrm{Categorical}\bigl(\pi_1(s_i),\pi_2(s_i),\dots\bigr)
      \]
      \[
        \mathbf y_i \mid z_i,\{\theta_h\} \sim
        \mathrm{Multinomial}\bigl(N_i,\;\theta_{z_i}\bigr).
      \]
      のように書き換えることができる。

\end{itemize}
\end{frame}

%----------------------------------------------------------------------
\begin{frame}{モデルの階層構造}
\[
\begin{aligned}
\Gamma_h &\sim H, \\
V_h &\sim\operatorname{Beta}(1,\lambda),\\
\theta_h &\sim\operatorname{Dirichlet}\!\Bigl(\tfrac{\gamma_0}{K}\mathbf 1_K\Bigr),
& & \\[4pt]
\pi_h(s) &= V_h\,K(s,\Gamma_h)\!\!\prod_{l<h}\!\bigl[1-V_lK(s,\Gamma_l)\bigr],\\
\pi(s) &= \sum_{h}\pi_h(s)\,\theta_h,\\
z_i \mid s_i &\sim\mathrm{Categorical}\bigl(\pi_1(s_i),\pi_2(s_i),\dots\bigr), \\
\mathbf y_i \mid z_i &\sim\mathrm{Multinomial}\!\bigl(N_i,\theta_{z_i}\bigr).
\end{aligned}
\]
\end{frame}

%----------------------------------------------------------------------
\begin{frame}{同時事後分布}
\[
\begin{aligned}
& p\!\bigl(\{\Gamma_h\},\{V_h\},\{\theta_h\},
           \mathbf z,\mathbf y \mid \mathbf s\bigr) \\[2pt]
&=
   \prod_{h=1}^{\infty}
     {%
       H(\Gamma_h)\;
       \mathrm{Beta}\!\bigl(V_h\mid 1,\lambda\bigr)\;
       \mathrm{Dir}\!\Bigl(\theta_h\mid\tfrac{\gamma_0}{K}\mathbf 1_K\Bigr)
     }
   \times
   \prod_{i=1}^{n}
     {%
       \mathrm{Cat}\!\bigl(z_i\mid\pi_1(s_i),\pi_2(s_i),\dots\bigr)
       \mathrm{Mult}\!\bigl(\mathbf y_i\mid N_i,\theta_{z_i}\bigr)
     } .
\end{aligned}
\]

\[
\pi_h(s) \;=\;
  V_h\,K(s,\Gamma_h)\,
  \prod_{l<h}\!\bigl[1-V_l K(s,\Gamma_l)\bigr],
\quad
\mathbf z=(z_1,\dots,z_n).
\]
\end{frame}

%----------------------------------------------------------------------
\begin{frame}{\(\boldsymbol{\theta}_h\) の完全条件付き分布は Dirichlet 分布になる}

$$
\begin{aligned}
P\left(\boldsymbol{\theta}_h \mid \mathbf{z}, \mathbf{y}\right) &\propto P\left(\boldsymbol{\theta}_h\right) \prod_{\hat{i}_i: z_i=h} P\left(\mathbf{y}_i \mid \boldsymbol{\theta}_h\right)
\\
& \propto \prod_{k=1}^K \theta_{h k}^{\frac{\pi}{K}-1+\sum_{n=i-h} y_{i k}} \\
& =\operatorname{Dirichlet}\left(\frac{\gamma_0}{K}+S_{h 1}, \ldots, \frac{\gamma_0}{K}+S_{h K}\right),
\end{aligned}
$$

$$
S_{hk}\;:=\;\sum_{i:z_i=h} y_{ik},
\qquad
S_{h\cdot}=\sum_{k}S_{hk}.
$$

\end{frame}

%----------------------------------------------------------------------
\begin{frame}{\(z_i\) の完全条件付き分布はカテゴリ分布になる}

\[
\begin{aligned}
P(z_i=h \mid \text{rest})
 &\;\propto\;
    P(z_i=h \mid \mathbf V,\Gamma,s_i)\;
    P(\mathbf y_i \mid z_i=h,\boldsymbol\theta_h) \\[4pt]
 &\;=\;
    \pi_h(s_i)\;
    \prod_{k=1}^{K}\theta_{hk}^{\,y_{ik}} .
\end{aligned}
\]

\vspace{2pt}
未規格化重みを
\[
w_{ih}:=\pi_h(s_i)\,
         \prod_{k}\theta_{hk}^{\,y_{ik}},\qquad h=1,2,\dots
\]
とおくと

\[
{%
   P(z_i=h \mid \text{rest})=
     \frac{w_{ih}}{\sum_{g}w_{ig}}
   \;\;\;(\text{カテゴリ分布})}.
\]

となる。
\end{frame}

%----------------------------------------------------------------------
\begin{frame}{V\(_h\) のサンプリング: Slice Sampling}

$V_h$のサンプリングは、Walker(2007)\cite{Walker2007-aj}のSlice samplingの手法に従って行う。

Walker(2007)\cite{Walker2007-aj}によれば、スライスのための補助変数\; \(u_i\sim\text{Uniform}(0,1)\) を置き,
\[
u_i < \pi_{z_i}(s_i)
      = V_{z_i} K(s_i,\Gamma_{z_i})
        \prod_{g<z_i}(1-V_g K(s_i,\Gamma_g)).
\]
の不等式によって、無限個のうち有限個のクラスタのみを選べば、
\[
\begin{aligned}
m_h &= \#\{i : z_i = h\},\\
r_h &= \#\{i : z_i  > h,\;
                u_i < V_h K(s_i,\Gamma_h)
                \prod_{g<h}(1-V_g K(s_i,\Gamma_g))\}.
\end{aligned}
\]

とおくことによって、
\[
V_h \mid \text{rest} \sim \text{Beta}(1+m_h,\; \lambda+r_h).
\]
が完全条件付き分布になることが知られている。

\end{frame}

%----------------------------------------------------------------------
\begin{frame}{位置パラメータ \(\Gamma_h\) はMH法で更新}
\[
\begin{aligned}
p(\Gamma_h \mid \text{rest})
   &\;\propto\;
     H(\Gamma_h)\,
     \prod_{i=1}^{n}
         P(z_i,u_i \mid \Gamma_h, \text{rest}) \\[4pt]
   &\;\propto\;
     H(\Gamma_h)\,
     {\prod_{i:z_i=h}\pi_h(s_i)}
     \;
     {\prod_{i:z_i>h,\;u_i<\pi_h(s_i)}
                     \bigl[1-\pi_h(s_i)\bigr]}
\\
   &\;\propto\;
     H(\Gamma_h)\,
     \prod_{i:z_i=h}K(s_i,\Gamma_h)\,
     \prod_{i:z_i>h}
       \bigl[1-K(s_i,\Gamma_h)\bigr]^{\mathbf 1\{u_i<\pi_h(s_i)\}}
\end{aligned}
\]

\vspace{4pt}
\begin{itemize}
\item 解析形は得られないため、ランダムウォーク Metropolis-Hastings法 を採用。
\item 提案分布: $\mathcal N(0,\sigma_\phi^{2}I_2)$
\end{itemize}

\end{frame}

\begin{frame}{実験}

Multinomial-KSBPモデルを実データに適用した。

共変量として、

\begin{itemize}
    \item 標高
    \item 傾斜角度
    \item それぞれ4つの産地からの距離
    \item 最寄りの川・湖からの距離
\end{itemize}

を用いた。

ハイパーパラメータは、

\begin{itemize}
    \item $\lambda=1$ (スティック長$V_h$のベータ事前分布の尺度パラメータ)
    \item $\gamma_0=0.1$ (比率$\theta_h$のディリクレ事前分布の集中度パラメータ)
\end{itemize}

と設定した。カーネルにはすべてガウスカーネルを用い、パラメータは適宜調整した。MCMCのイテレーション数は2000とした。

\end{frame}

\begin{frame}{結果}

例として早期・早々期の神津島、信州の結果を示す。


\begin{figure}
    \centering
    \begin{subfigure}{0.49\textwidth}
        \centering
        \includegraphics[width=\textwidth]{fig/image (4).png}
        \caption{$\pi_k(s)$の事後平均: 早期・早々期、神津島}
        \label{fig:pointprocess1}
    \end{subfigure}
    \hfill
    \begin{subfigure}{0.49\textwidth}
        \centering
        \includegraphics[width=\textwidth]{fig/image (5).png}
        \caption{$\pi_k(s)$の事後平均: 早期・早々期、信州}
        \label{fig:pointprocess2}
    \end{subfigure}
    \label{fig:both-predictions}
\end{figure}

\end{frame}

\begin{frame}{結果}

観測データの少ない箇所の事後標準偏差が大きく、推定の不確実性を評価することができた。

\begin{figure}
    \centering
    \includegraphics[width=0.8\linewidth]{fig/image (6).png}
    \caption{$\pi_k(s)$の事後標準偏差}
    \label{fig:enter-label}
\end{figure}
\end{frame}

\begin{frame}{考察}

\begin{itemize}
    \item 不確実性を含め、産地構成比の推定を行うことができた。
    \item カーネルの選択も含め、ハイパーパラメータの幅広い探索を行うことが必要。(現在はガウスカーネルのみ)
    \item \textbf{観測データがほとんどない地点の推定}をどのように行ってほしいかについて検討が必要
    (NW推定量は「最近傍の比率をそのままコピー」するような挙動に対し、KSBPは「カーネルの距離に応じて不確実性が増大」する挙動)
    \item 逆に、NW推定量と同じような挙動を示すようなNPベイズモデルを作れないか:
    $$\mathbb{E}[\pi_k(x) \mid \mathcal{D}] =
\frac{ \frac{\gamma_0}{K} + \sum_{i=1}^n K(x, x_i)\, y_{ik} }{ \gamma_0 + \sum_{i=1}^n K(x, x_i)\, N_i }$$
    \item Dirichlet Processと並んでノンパラメトリックベイズ法の代表的な方法であるGaussian Processを元にしたモデルも実装し、推定結果を比較したい。

\end{itemize}

\end{frame}



\section{まとめと今後の課題}

\begin{frame}
{\Large 目次}
 \tableofcontents[currentsection]
\end{frame}

\begin{frame}{まとめと今後の課題}
\begin{itemize}
    \item 遺跡の存在確率のモデルとして非斉次ポアソン過程、黒曜石の産地構成比のモデルとして2つのノンパラメトリック回帰モデル(Nadaraya-Watson推定量、Multinomial Kernel Stick-Breaking Process)を実装した。
    \item 産地構成比についてもベイズ化することで、遺跡の存在確率と産地構成比を互いに独立ではなく依存させた形でモデル化する方法もこれから考えられる。
    \item 産地構成比への共変量の効果のうち、線形のものと非線形のものを分離するためのモデルを考えられないか:セミパラメトリックな拡張を考える

\end{itemize}

\end{frame}


\begin{frame}
\frametitle{データとコード}

\url{https://github.com/ARUOHTA/inhomogeneous_poisson_process/blob/main/bayesian_statistics}

\end{frame}

\bibliographystyle{unsrt} %参考文献出力スタイル
\bibliography{reference}

\end{document}
