\documentclass[a4paper, 11pt]{article}

% XeLaTeX + 日本語設定
\usepackage{fontspec}
\setmainfont{TeX Gyre Pagella}
\usepackage{xeCJK}
\setCJKmainfont{Noto Serif CJK JP}

% ページ設定
\usepackage{geometry}
\geometry{margin=2.5cm}

% 数式
\usepackage{amsmath, amssymb, amsthm}
\usepackage{bm}

% 定理環境
\theoremstyle{definition}
\newtheorem{definition}{定義}[section]
\newtheorem{theorem}{定理}[section]
\newtheorem{proposition}{命題}[section]

% 図表
\usepackage{graphicx}

% リンク
\usepackage{hyperref}

% タイトル
\title{マーク付き点過程による黒曜石産地構成比のモデル化}
\author{太田 阿留}
\date{2024年12月20日}

\begin{document}

\maketitle

\begin{abstract}
  本ノートでは、非斉次ポアソン過程(IPP)による遺跡存在確率モデルと、
  NNGPによる産地構成比モデルを統合したマーク付き点過程モデルの理論的定式化を行う。
\end{abstract}

\tableofcontents

\section{導入}
% TODO

\input{sections/sec2}
\section{階層ベイズモデルの全体構造}

前節では,遺跡位置に対する時空間ポアソン点過程と,各遺跡のマーク(産地構成比)に対する多項ロジットモデルを,それぞれ個別に定式化した.本節では,これらを一つの階層ベイズモデルとして統合し,事前分布を含めた全体構造を明示する.

まず,パラメータと潜在変数の集合を
\[
\Theta
=
\Bigl\{
\lambda^*,\,
\boldsymbol{\beta}_{\mathrm{int}},
\{\boldsymbol{\beta}_k\}_{k=1}^{K-1},
u_{\mathrm{int}}(\cdot,\cdot),
\{u_k(\cdot,\cdot)\}_{k=1}^{K-1}
\Bigr\}
\]
と書く.ここで,$u_{\mathrm{int}}$ および $u_k$ は時空間上のガウス過程(もしくはその NNGP 近似)とみなす.実際の計算では,観測点上での有限次元ベクトルとして扱うので,後で離散化した形に書き直す.

観測データは,遺跡の位置と時期の集合 $X = \{(s_i,t_i)\}_{i=1}^{n_X}$ と,各遺跡でのマーク $\boldsymbol{y}_i$ の集合
\[
\mathcal{Y} = \{\boldsymbol{y}_i\}_{i=1}^{n_X}
\]
からなる.

階層モデルの構造は,概念的には次のように書ける:

\textbf{1.} 上位階層(ハイパーパラメータ)
\[
\lambda^* \sim \mathrm{Ga}(m_0,r_0),\qquad
\boldsymbol{\beta}_{\mathrm{int}} \sim \mathcal{N}(\boldsymbol{b}_0^{\mathrm{int}}, B_0^{\mathrm{int}}),
\]
\[
\boldsymbol{\beta}_k \sim \mathcal{N}(\boldsymbol{b}_0^{(k)}, B_0^{(k)}),\quad k=1,\dots,K-1.
\]

\textbf{2.} 潜在ガウス場(NNGP 近似)

例えば,遺跡の潜在強度場について
\[
\boldsymbol{u}_{\mathrm{int}}
=
\bigl(u_{\mathrm{int}}(s_1,t_1),\dots,u_{\mathrm{int}}(s_{n_X},t_{n_X})\bigr)^\top
\sim
\mathcal{N}\bigl(\boldsymbol{0},\, Q_{\mathrm{int}}^{-1}\bigr),
\]
産地 $k$ に対する潜在場について
\[
\boldsymbol{u}_k
=
\bigl(u_k(s_1,t_1),\dots,u_k(s_{n_X},t_{n_X})\bigr)^\top
\sim
\mathcal{N}\bigl(\boldsymbol{0},\, Q_k^{-1}\bigr),
\quad k=1,\dots,K-1,
\]
とする.ここで $Q_{\mathrm{int}}, Q_k$ は NNGP によって与えられるスパースな精度行列であり,距離に基づく共分散構造を近似しているものとする(ハイパーパラメータはここでは既知とみなす).

\textbf{3.} 遺跡位置の生成(点過程部分)

強度関数
\[
\lambda(s,t)
=
\lambda^* \, q(s,t),
\qquad
q(s,t) =
\frac{\exp\{\eta_{\mathrm{int}}(s,t)\}}{1+\exp\{\eta_{\mathrm{int}}(s,t)\}},
\]
\[
\eta_{\mathrm{int}}(s,t)
=
\boldsymbol{w}_{\mathrm{int}}(s,t)^\top \boldsymbol{\beta}_{\mathrm{int}}
+ u_{\mathrm{int}}(s,t),
\]
に従い,
\[
X \mid \lambda^*, \boldsymbol{\beta}_{\mathrm{int}}, u_{\mathrm{int}}
\sim \mathrm{IPP}(\lambda).
\]

\textbf{4.} マークの生成(多項ロジット部分)

各遺跡 $(s_i,t_i)$ について,
\[
\eta_k(s_i,t_i)
=
\boldsymbol{w}_z(s_i,t_i)^\top \boldsymbol{\beta}_k
+ u_k(s_i,t_i),\quad k=1,\dots,K-1,
\]
\[
\pi_k(s_i,t_i)
=
\frac{\exp\{\eta_k(s_i,t_i)\}}{1+\sum_{k'=1}^{K-1}\exp\{\eta_{k'}(s_i,t_i)\}},
\quad
\pi_K(s_i,t_i)
=
\frac{1}{1+\sum_{k'=1}^{K-1}\exp\{\eta_{k'}(s_i,t_i)\}},
\]
と定義し,
\[
\boldsymbol{y}_i
\mid
N_i,\,
\{\boldsymbol{\beta}_k\},\,
\{u_k\}
\sim
\mathrm{Multinomial}\bigl(N_i,\ \boldsymbol{\pi}(s_i,t_i)\bigr),
\qquad
i=1,\dots, n_X,
\]
とする.

このように,点過程部分とマーク部分は,それぞれに固有の潜在場 $u_{\mathrm{int}}$ と $\{u_k\}$ を持ちつつ,時空間上の同じ位置 $(s_i,t_i)$ で評価される.後者は NNGP によって効率的に扱われる.

\input{sections/sec4}
\input{sections/sec5}
\section{完全条件付き分布:マーク部分}

最後に,マーク(産地構成比)の NNGP 多項ロジット部分について,Polya--Gamma 拡張の形を用いた完全条件付き分布をまとめる.ここでは構造を明示することを主眼とし,導出は点過程部分と同型であるため要所のみ書く.

\subsection{多項ロジットモデルの Polya--Gamma 形式}

各遺跡 $i=1,\dots,n_X$ と各カテゴリ $k=1,\dots,K-1$ について,
\[
\eta_{ik} = \eta_k(s_i,t_i)
= \boldsymbol{w}_z(s_i,t_i)^\top \boldsymbol{\beta}_k + u_k(s_i,t_i),
\]
\[
\pi_{ik}
=
\frac{\exp\{\eta_{ik}\}}
{1+\sum_{k'=1}^{K-1} \exp\{\eta_{ik'}\}},
\quad
\pi_{iK}
=
\frac{1}{1+\sum_{k'=1}^{K-1} \exp\{\eta_{ik'}\}},
\]
とおいた.多項尤度は
\[
p(\boldsymbol{y}_i \mid N_i,\{\eta_{ik}\})
\propto
\frac{\prod_{k=1}^K \exp\{y_{ik}\eta_{ik}\}}
{\Bigl(\sum_{k'=1}^K \exp\{\eta_{ik'}\}\Bigr)^{N_i}},
\]
である.

基準カテゴリ $K$ を固定すると,多項尤度は次のように変形できる:
\[
\begin{aligned}
p(\boldsymbol{y}_i \mid N_i,\{\eta_{ik}\})
&\propto
\prod_{k=1}^{K-1} \exp\{y_{ik}\eta_{ik}\}
\cdot
\left(1 + \sum_{\ell=1}^{K-1} \exp\{\eta_{i\ell}\}\right)^{-N_i}.
\end{aligned}
\]
ここで,各カテゴリ $k=1,\dots,K-1$ に対して独立に Polya--Gamma 恒等式を適用する.
具体的には,$a = y_{ik}$, $b = N_i$ として
\[
\frac{(e^{\eta_{ik}})^{y_{ik}}}{(1+e^{\eta_{ik}})^{N_i}}
=
2^{-N_i} \exp\{\tilde{\kappa}_{ik} \eta_{ik}\}
\int_0^\infty \exp\left\{-\frac{\xi_{ik} \eta_{ik}^2}{2}\right\}
p(\xi_{ik} \mid N_i, 0)\, d\xi_{ik},
\]
\[
\tilde{\kappa}_{ik} = y_{ik} - \frac{N_i}{2},
\qquad
\xi_{ik} \sim \mathrm{PG}(N_i, 0)
\]
を用いる.$\xi_{ik}$ を潜在変数として導入し,条件付けると,
\[
p(\boldsymbol{y}_i \mid N_i, \{\eta_{ik}\}, \{\xi_{ik}\})
\propto
\prod_{k=1}^{K-1}
\exp\left\{
\tilde{\kappa}_{ik} \eta_{ik} - \frac{\xi_{ik} \eta_{ik}^2}{2}
\right\}.
\]
点過程部分と同様に二乗完成を行うと,
\[
\tilde{\kappa}_{ik} \eta_{ik} - \frac{\xi_{ik} \eta_{ik}^2}{2}
=
-\frac{\xi_{ik}}{2}
\left(\eta_{ik} - \frac{\tilde{\kappa}_{ik}}{\xi_{ik}}\right)^2
+ \frac{\tilde{\kappa}_{ik}^2}{2\xi_{ik}},
\]
となる.後者は $\eta_{ik}$ に依存しないため無視できる.

全遺跡についてまとめると,
\[
p(\mathcal{Y} \mid \{\boldsymbol{\beta}_k\},\{u_k\},\Xi)
\propto
\prod_{k=1}^{K-1}
\exp\left\{
-\frac12
(\boldsymbol{\eta}_k - \boldsymbol{z}_k)^\top
\Omega_k
(\boldsymbol{\eta}_k - \boldsymbol{z}_k)
\right\},
\]
という形に書ける.ここで,
\begin{itemize}
\item $\boldsymbol{\eta}_k = (\eta_{1k},\dots,\eta_{n_X k})^\top$
\item $\Omega_k = \mathrm{diag}(\xi_{1k},\dots,\xi_{n_X k})$
\item $\boldsymbol{z}_k = \left(\dfrac{\tilde{\kappa}_{1k}}{\xi_{1k}},\dots,\dfrac{\tilde{\kappa}_{n_X k}}{\xi_{n_X k}}\right)^\top$
\item $\tilde{\kappa}_{ik} = y_{ik} - \dfrac{N_i}{2}$
\end{itemize}
である.

\subsection{\texorpdfstring{$(\boldsymbol{\beta}_k, \boldsymbol{u}_k)$}{(β\_k, u\_k)} の完全条件付き分布}

カテゴリ $k$ 固定で考える.$\boldsymbol{\eta}_k = W_z \boldsymbol{\beta}_k + \boldsymbol{u}_k$ と書く.ここで $W_z$ は,$\boldsymbol{w}_z(s_i,t_i)^\top$ を行ベクトルに持つ設計行列である.事前分布は
\[
\boldsymbol{\beta}_k \sim \mathcal{N}(\boldsymbol{b}_0^{(k)}, B_0^{(k)}),
\qquad
\boldsymbol{u}_k \sim \mathcal{N}(\boldsymbol{0}, Q_k^{-1})
\]
とする($Q_k$ は NNGP の精度行列).

点過程の場合と全く同様に,
\[
\theta_k =
\begin{pmatrix}
\boldsymbol{\beta}_k \\
\boldsymbol{u}_k
\end{pmatrix},
\quad
\mu_{0,k} =
\begin{pmatrix}
\boldsymbol{b}_0^{(k)} \\
\boldsymbol{0}
\end{pmatrix},
\quad
R_{0,k} =
\begin{pmatrix}
(B_0^{(k)})^{-1} & 0 \\
0 & Q_k
\end{pmatrix},
\quad
H_z = [\, W_z\ \ I_{n_X} \,],
\]
と置くと,$\theta_k$ に関する(比例)事後密度は
\[
\log p(\theta_k \mid \cdots)
=
-\frac12 (H_z\theta_k - \boldsymbol{z}_k)^\top \Omega_k (H_z\theta_k - \boldsymbol{z}_k)
-\frac12 (\theta_k - \mu_{0,k})^\top R_{0,k}(\theta_k - \mu_{0,k})
+ \text{const},
\]
であり,二乗完成の結果,
\[
\boxed{
\theta_k \mid \Xi, \mathcal{Y}, X
\sim
\mathcal{N}\bigl(\boldsymbol{m}_k, V_k\bigr),
}
\]
\[
V_k^{-1} = H_z^\top \Omega_k H_z + R_{0,k},
\qquad
\boldsymbol{m}_k = V_k \bigl(H_z^\top \Omega_k \boldsymbol{z}_k + R_{0,k}\mu_{0,k}\bigr).
\]

これを $k=1,\dots,K-1$ について繰り返すことで,全てのマーク・カテゴリに対する $(\boldsymbol{\beta}_k, \boldsymbol{u}_k)$ を更新できる.

\subsection{Polya--Gamma 変数 \texorpdfstring{$\xi_{ik}$}{ξ\_ik} の完全条件付き分布}

各 $(i,k)$ について,出てくる形は
\[
\exp\Bigl\{ \tilde{\kappa}_{ik}\eta_{ik} - \frac{\xi_{ik}\eta_{ik}^2}{2}\Bigr\}p(\xi_{ik}),
\]
という形になっているので,点過程部分と同じ議論から,
\[
\boxed{
\xi_{ik} \mid \eta_{ik}, N_i
\sim
\mathrm{PG}\bigl(N_i,\ \eta_{ik}\bigr).
}
\]
ここで $N_i$ は遺跡 $i$ での総出土数である.

\section{ギブスサンプリングアルゴリズム(まとめ)}

以上をまとめると,本 joint モデルのギブスサンプリングは概略次のようになる:

\textbf{1.} 初期値
\[
\lambda^{*(0)},\ \boldsymbol{\beta}_{\mathrm{int}}^{(0)},\ \boldsymbol{u}_{\mathrm{int}}^{(0)},
\ \{\boldsymbol{\beta}_k^{(0)},\boldsymbol{u}_k^{(0)}\}_{k=1}^{K-1},
\ \boldsymbol{\omega}^{(0)},\ \Xi^{(0)},\ U^{(0)}
\]
を適当に設定する.

\textbf{2.} 各反復 $\tau = 1,\dots,T$ で:

\textbf{(a) 偽不在の更新}
\[
U^{(\tau)} \sim \mathrm{IPP}\bigl(\lambda^{*(\tau-1)}(1-q^{(\tau-1)})\bigr)
\]
を Poisson thinning によってサンプルする.

\textbf{(b) 点過程側 Polya--Gamma の更新}
\[
\omega_i^{(\tau)}
\sim
\mathrm{PG}\bigl(1,\ \eta_{\mathrm{int},i}^{(\tau-1)}\bigr),
\quad i=1,\dots,n_X+n_U^{(\tau)}.
\]

\textbf{(c) $(\boldsymbol{\beta}_{\mathrm{int}},\boldsymbol{u}_{\mathrm{int}})$ の更新}
\[
\theta_{\mathrm{int}}^{(\tau)}
=
\begin{pmatrix}
\boldsymbol{\beta}_{\mathrm{int}}^{(\tau)} \\
\boldsymbol{u}_{\mathrm{int}}^{(\tau)}
\end{pmatrix}
\sim
\mathcal{N}\bigl(\boldsymbol{m}_{\mathrm{int}}, V_{\mathrm{int}}\bigr),
\]
ただし $\boldsymbol{m}_{\mathrm{int}}, V_{\mathrm{int}}$ は $\boldsymbol{\omega}^{(\tau)}, X, U^{(\tau)}$ から計算する.

\textbf{(d) $\lambda^*$ の更新}
\[
\lambda^{*(\tau)} \sim \mathrm{Ga}\bigl(m_0 + n^{(\tau)},\ r_0 + |\mathcal{D}|\bigr),
\]
ここで $n^{(\tau)} = n_X + n_U^{(\tau)}$.

\textbf{(e) マーク側 Polya--Gamma の更新}

各 $i,k$ について
\[
\xi_{ik}^{(\tau)} \sim \mathrm{PG}\bigl(N_i,\ \eta_{ik}^{(\tau-1)}\bigr).
\]

\textbf{(f) 各カテゴリ $k$ の $(\boldsymbol{\beta}_k,\boldsymbol{u}_k)$ の更新}
\[
\theta_k^{(\tau)}
=
\begin{pmatrix}
\boldsymbol{\beta}_k^{(\tau)} \\
\boldsymbol{u}_k^{(\tau)}
\end{pmatrix}
\sim
\mathcal{N}\bigl(\boldsymbol{m}_k, V_k\bigr),
\]
ただし $\boldsymbol{m}_k, V_k$ は $\Xi^{(\tau)}, \mathcal{Y}, X$ から計算する.

\section{時間依存構造を持つ DGP の導入}

\subsection{基本的な考え方}

これまでの定式化では、「時期 $t$ はあくまで共変量の 1 つ」であり、空間ランダム効果 $u_{\mathrm{int}}(s)$ や $u_k(s)$ は「空間だけの GP」として独立に定義していた。

しかし実際には、「早期の分布と中期の分布には連続性がある」「時間が近いほど似た空間パターンをとる」と考えるのが自然である。そこで、以下のような構造を導入する:

\begin{itemize}
\item 時間 $t=1,\dots,T$ ごとに、空間ランダム効果の場 $u_{\mathrm{int}}^{(t)}(s)$, $u_k^{(t)}(s)$ を定義する。
\item これらの場どうしに、時間方向の相関を持たせる(近い時点ほど強く相関する)。
\end{itemize}

このような「時間インデックス付き GP の族」をまとめて 1 つの GP と見る立場を、ここでは DGP と呼ぶことにする。

形式的には、領域 $\mathcal{D}\subset\mathbb{R}^2$ と離散時点 $t\in\{1,\dots,T\}$ を合わせた積空間
\[
\mathcal{D}_\mathrm{st} = \mathcal{D}\times\{1,\dots,T\}
\]
上の GP
\[
u_{\mathrm{int}}(s,t),\quad u_k(s,t)
\]
を定義し、共分散構造で時間依存を表現する。

\subsection{強度側 DGP の具体形}

IPP の強度側で用いていたロジット線形予測子は、時間を明示すると
\[
\eta_{\mathrm{int}}(s,t)
=
\boldsymbol{w}_{\mathrm{int}}(s,t)^\top\boldsymbol{\beta}_{\mathrm{int}}
+
u_{\mathrm{int}}(s,t)
\]
であり、存在確率は
\[
q(s,t)
=
\frac{\exp\{\eta_{\mathrm{int}}(s,t)\}}
{1+\exp\{\eta_{\mathrm{int}}(s,t)\}}.
\]

ここで、$u_{\mathrm{int}}(s,t)$ に対し、空間--時間 GP を
\[
u_{\mathrm{int}}(\cdot,\cdot)
\sim
\mathrm{GP}\bigl(0,\ C_{\mathrm{int}}((s,t),(s',t'))\bigr)
\]
として定義する。もっとも単純で扱いやすいのは「空間と時間の直積(セパラブル)共分散」であり、
\[
C_{\mathrm{int}}((s,t),(s',t'))
=
\sigma_{\mathrm{int}}^2\,
C_{\mathrm{space}}^{\mathrm{int}}(s,s';\phi_{\mathrm{int}})
\,
C_{\mathrm{time}}^{\mathrm{int}}(t,t';\rho_{\mathrm{int}})
\]
のように書く。ここで

\begin{itemize}
\item $C_{\mathrm{space}}^{\mathrm{int}}(s,s';\phi_{\mathrm{int}})$ は空間距離に依存する共分散(例:指数型, Mat\'ern など)
\item $C_{\mathrm{time}}^{\mathrm{int}}(t,t';\rho_{\mathrm{int}}) = \rho_{\mathrm{int}}^{|t-t'|}$ のような AR(1) 型の時間共分散
\item $\sigma_{\mathrm{int}}^2$ は分散スケール
\end{itemize}

とする。

このとき、観測・偽不在の点を合わせた空間--時間位置
\[
(s_1,t_1),\dots,(s_n,t_n)
\]
で評価したベクトル
\[
\boldsymbol{u}_{\mathrm{int}}
=
\bigl(u_{\mathrm{int}}(s_1,t_1),\dots,u_{\mathrm{int}}(s_n,t_n)\bigr)^\top
\]
は
\[
\boldsymbol{u}_{\mathrm{int}}
\sim
\mathcal{N}\bigl(\boldsymbol{0},\ \Sigma_{\mathrm{int}}\bigr)
\]
に従い,$\Sigma_{\mathrm{int}}$ は上記カーネルから構成される共分散行列になる。

セパラブル構造を採用し,時点ごとの空間位置集合が同じ(あるいはグリッド上)であれば,
\[
\Sigma_{\mathrm{int}} = \Sigma_\mathrm{time}^{\mathrm{int}}\ \otimes\ \Sigma_\mathrm{space}^{\mathrm{int}}
\]
という Kronecker 積の形を持つ。ここで

\begin{itemize}
\item $\Sigma_\mathrm{time}^{\mathrm{int}}$ は $T\times T$ の時間共分散行列
\item $\Sigma_\mathrm{space}^{\mathrm{int}}$ は空間位置についての共分散行列
\end{itemize}

である。セパラブル共分散の精度行列は
\[
Q_{\mathrm{int}}
=
\Sigma_{\mathrm{int}}^{-1}
=
(\Sigma_\mathrm{time}^{\mathrm{int}})^{-1} \otimes (\Sigma_\mathrm{space}^{\mathrm{int}})^{-1}
=
Q_\mathrm{time}^{\mathrm{int}}\ \otimes\ Q_\mathrm{space}^{\mathrm{int}}
\]
となる。NNGP を使うときは、空間側の精度行列 $Q_\mathrm{space}^{\mathrm{int}}$ を NNGP による疎な精度行列で近似する。

いずれにせよ、「$\boldsymbol{u}_{\mathrm{int}}$ は多次元正規であり、事前分布は
\[
\boldsymbol{u}_{\mathrm{int}} \sim \mathcal{N}(\boldsymbol{0}, Q_{\mathrm{int}}^{-1})
\]
という形に保たれる」ことが重要である。

\subsection{マーク側 DGP の具体形}

マーク側の線形予測子も,時間を明示すると
\[
\eta_k(s,t)
=
\boldsymbol{w}_z(s,t)^\top\boldsymbol{\beta}_k
+
u_k(s,t),\quad k=1,\dots,K-1,
\]
\[
\pi_k(s,t)
=
\frac{\exp\{\eta_k(s,t)\}}
{\sum_{k'=1}^K\exp\{\eta_{k'}(s,t)\}},
\quad
\eta_K(s,t)\equiv 0
\]
であった。

ここでも同様に,「カテゴリ $k$ ごとに空間--時間 GP を定義」する:
\[
u_k(\cdot,\cdot)
\sim
\mathrm{GP}\bigl(0,\ C_k((s,t),(s',t'))\bigr),
\]
\[
C_k((s,t),(s',t'))
=
\sigma_k^2\,
C_{\mathrm{space}}^{(k)}(s,s';\phi_k)
\,
C_{\mathrm{time}}^{(k)}(t,t';\rho_k),
\]
とする。簡単のため,強度側と同じ形式(セパラブル,時間 AR(1))を用いる。

各時期 $t$ における観測遺跡 $i=1,\dots,n_{X}^{(t)}$ の空間位置 $s_i^{(t)}$ を集め,すべての時期をまとめたインデックスを $i=1,\dots,n_X$ と書くと,
\[
\boldsymbol{u}_k
=
\bigl(u_k(s_1,t_1),\dots,u_k(s_{n_X},t_{n_X})\bigr)^\top
\sim
\mathcal{N}\bigl(\boldsymbol{0}, Q_k^{-1}\bigr),
\]
という形で表現できる。ここで $Q_k$ は NNGP+時間共分散から構成した精度行列である。

このようにして、IPP 強度側の $u_{\mathrm{int}}(s,t)$ と、マーク側の $u_k(s,t)$ の両方に対して、時間依存を持つ DGP 構造を入れたことになる。

\input{sections/sec9}

\end{document}
