\section{階層ベイズモデルの全体構造}

前節では,遺跡位置に対する時空間ポアソン点過程と,各遺跡のマーク(産地構成比)に対する多項ロジットモデルを,それぞれ個別に定式化した.本節では,これらを一つの階層ベイズモデルとして統合し,事前分布を含めた全体構造を明示する.

まず,パラメータと潜在変数の集合を
\[
\Theta
=
\Bigl\{
\lambda^*,\,
\boldsymbol{\beta}_{\mathrm{int}},
\{\boldsymbol{\beta}_k\}_{k=1}^{K-1},
u_{\mathrm{int}}(\cdot,\cdot),
\{u_k(\cdot,\cdot)\}_{k=1}^{K-1}
\Bigr\}
\]
と書く.ここで,$u_{\mathrm{int}}$ および $u_k$ は時空間上のガウス過程(もしくはその NNGP 近似)とみなす.実際の計算では,観測点上での有限次元ベクトルとして扱うので,後で離散化した形に書き直す.

観測データは,遺跡の位置と時期の集合 $X = \{(s_i,t_i)\}_{i=1}^{n_X}$ と,各遺跡でのマーク $\boldsymbol{y}_i$ の集合
\[
\mathcal{Y} = \{\boldsymbol{y}_i\}_{i=1}^{n_X}
\]
からなる.

階層モデルの構造は,概念的には次のように書ける:

\textbf{1.} 上位階層(ハイパーパラメータ)
\[
\lambda^* \sim \mathrm{Ga}(m_0,r_0),\qquad
\boldsymbol{\beta}_{\mathrm{int}} \sim \mathcal{N}(\boldsymbol{b}_0^{\mathrm{int}}, B_0^{\mathrm{int}}),
\]
\[
\boldsymbol{\beta}_k \sim \mathcal{N}(\boldsymbol{b}_0^{(k)}, B_0^{(k)}),\quad k=1,\dots,K-1.
\]

\textbf{2.} 潜在ガウス場(NNGP 近似)

例えば,遺跡の潜在強度場について
\[
\boldsymbol{u}_{\mathrm{int}}
=
\bigl(u_{\mathrm{int}}(s_1,t_1),\dots,u_{\mathrm{int}}(s_{n_X},t_{n_X})\bigr)^\top
\sim
\mathcal{N}\bigl(\boldsymbol{0},\, Q_{\mathrm{int}}^{-1}\bigr),
\]
産地 $k$ に対する潜在場について
\[
\boldsymbol{u}_k
=
\bigl(u_k(s_1,t_1),\dots,u_k(s_{n_X},t_{n_X})\bigr)^\top
\sim
\mathcal{N}\bigl(\boldsymbol{0},\, Q_k^{-1}\bigr),
\quad k=1,\dots,K-1,
\]
とする.ここで $Q_{\mathrm{int}}, Q_k$ は NNGP によって与えられるスパースな精度行列であり,距離に基づく共分散構造を近似しているものとする(ハイパーパラメータはここでは既知とみなす).

\textbf{3.} 遺跡位置の生成(点過程部分)

強度関数
\[
\lambda(s,t)
=
\lambda^* \, q(s,t),
\qquad
q(s,t) =
\frac{\exp\{\eta_{\mathrm{int}}(s,t)\}}{1+\exp\{\eta_{\mathrm{int}}(s,t)\}},
\]
\[
\eta_{\mathrm{int}}(s,t)
=
\boldsymbol{w}_{\mathrm{int}}(s,t)^\top \boldsymbol{\beta}_{\mathrm{int}}
+ u_{\mathrm{int}}(s,t),
\]
に従い,
\[
X \mid \lambda^*, \boldsymbol{\beta}_{\mathrm{int}}, u_{\mathrm{int}}
\sim \mathrm{IPP}(\lambda).
\]

\textbf{4.} マークの生成(多項ロジット部分)

各遺跡 $(s_i,t_i)$ について,
\[
\eta_k(s_i,t_i)
=
\boldsymbol{w}_z(s_i,t_i)^\top \boldsymbol{\beta}_k
+ u_k(s_i,t_i),\quad k=1,\dots,K-1,
\]
\[
\pi_k(s_i,t_i)
=
\frac{\exp\{\eta_k(s_i,t_i)\}}{1+\sum_{k'=1}^{K-1}\exp\{\eta_{k'}(s_i,t_i)\}},
\quad
\pi_K(s_i,t_i)
=
\frac{1}{1+\sum_{k'=1}^{K-1}\exp\{\eta_{k'}(s_i,t_i)\}},
\]
と定義し,
\[
\boldsymbol{y}_i
\mid
N_i,\,
\{\boldsymbol{\beta}_k\},\,
\{u_k\}
\sim
\mathrm{Multinomial}\bigl(N_i,\ \boldsymbol{\pi}(s_i,t_i)\bigr),
\qquad
i=1,\dots, n_X,
\]
とする.

このように,点過程部分とマーク部分は,それぞれに固有の潜在場 $u_{\mathrm{int}}$ と $\{u_k\}$ を持ちつつ,時空間上の同じ位置 $(s_i,t_i)$ で評価される.後者は NNGP によって効率的に扱われる.
