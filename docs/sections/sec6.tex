\section{完全条件付き分布:マーク部分}

最後に,マーク(産地構成比)の NNGP 多項ロジット部分について,Polya--Gamma 拡張の形を用いた完全条件付き分布をまとめる.ここでは構造を明示することを主眼とし,導出は点過程部分と同型であるため要所のみ書く.

\subsection{多項ロジットモデルの Polya--Gamma 形式}

各遺跡 $i=1,\dots,n_X$ と各カテゴリ $k=1,\dots,K-1$ について,
\[
\eta_{ik} = \eta_k(s_i,t_i)
= \boldsymbol{w}_z(s_i,t_i)^\top \boldsymbol{\beta}_k + u_k(s_i,t_i),
\]
\[
\pi_{ik}
=
\frac{\exp\{\eta_{ik}\}}
{1+\sum_{k'=1}^{K-1} \exp\{\eta_{ik'}\}},
\quad
\pi_{iK}
=
\frac{1}{1+\sum_{k'=1}^{K-1} \exp\{\eta_{ik'}\}},
\]
とおいた.多項尤度は
\[
p(\boldsymbol{y}_i \mid N_i,\{\eta_{ik}\})
\propto
\frac{\prod_{k=1}^K \exp\{y_{ik}\eta_{ik}\}}
{\Bigl(\sum_{k'=1}^K \exp\{\eta_{ik'}\}\Bigr)^{N_i}},
\]
である.

基準カテゴリ $K$ を固定すると,多項尤度は次のように変形できる:
\[
\begin{aligned}
p(\boldsymbol{y}_i \mid N_i,\{\eta_{ik}\})
&\propto
\prod_{k=1}^{K-1} \exp\{y_{ik}\eta_{ik}\}
\cdot
\left(1 + \sum_{\ell=1}^{K-1} \exp\{\eta_{i\ell}\}\right)^{-N_i}.
\end{aligned}
\]
ここで,各カテゴリ $k=1,\dots,K-1$ に対して独立に Polya--Gamma 恒等式を適用する.
具体的には,$a = y_{ik}$, $b = N_i$ として
\[
\frac{(e^{\eta_{ik}})^{y_{ik}}}{(1+e^{\eta_{ik}})^{N_i}}
=
2^{-N_i} \exp\{\tilde{\kappa}_{ik} \eta_{ik}\}
\int_0^\infty \exp\left\{-\frac{\xi_{ik} \eta_{ik}^2}{2}\right\}
p(\xi_{ik} \mid N_i, 0)\, d\xi_{ik},
\]
\[
\tilde{\kappa}_{ik} = y_{ik} - \frac{N_i}{2},
\qquad
\xi_{ik} \sim \mathrm{PG}(N_i, 0)
\]
を用いる.$\xi_{ik}$ を潜在変数として導入し,条件付けると,
\[
p(\boldsymbol{y}_i \mid N_i, \{\eta_{ik}\}, \{\xi_{ik}\})
\propto
\prod_{k=1}^{K-1}
\exp\left\{
\tilde{\kappa}_{ik} \eta_{ik} - \frac{\xi_{ik} \eta_{ik}^2}{2}
\right\}.
\]
点過程部分と同様に二乗完成を行うと,
\[
\tilde{\kappa}_{ik} \eta_{ik} - \frac{\xi_{ik} \eta_{ik}^2}{2}
=
-\frac{\xi_{ik}}{2}
\left(\eta_{ik} - \frac{\tilde{\kappa}_{ik}}{\xi_{ik}}\right)^2
+ \frac{\tilde{\kappa}_{ik}^2}{2\xi_{ik}},
\]
となる.後者は $\eta_{ik}$ に依存しないため無視できる.

全遺跡についてまとめると,
\[
p(\mathcal{Y} \mid \{\boldsymbol{\beta}_k\},\{u_k\},\Xi)
\propto
\prod_{k=1}^{K-1}
\exp\left\{
-\frac12
(\boldsymbol{\eta}_k - \boldsymbol{z}_k)^\top
\Omega_k
(\boldsymbol{\eta}_k - \boldsymbol{z}_k)
\right\},
\]
という形に書ける.ここで,
\begin{itemize}
\item $\boldsymbol{\eta}_k = (\eta_{1k},\dots,\eta_{n_X k})^\top$
\item $\Omega_k = \mathrm{diag}(\xi_{1k},\dots,\xi_{n_X k})$
\item $\boldsymbol{z}_k = \left(\dfrac{\tilde{\kappa}_{1k}}{\xi_{1k}},\dots,\dfrac{\tilde{\kappa}_{n_X k}}{\xi_{n_X k}}\right)^\top$
\item $\tilde{\kappa}_{ik} = y_{ik} - \dfrac{N_i}{2}$
\end{itemize}
である.

\subsection{\texorpdfstring{$(\boldsymbol{\beta}_k, \boldsymbol{u}_k)$}{(β\_k, u\_k)} の完全条件付き分布}

カテゴリ $k$ 固定で考える.$\boldsymbol{\eta}_k = W_z \boldsymbol{\beta}_k + \boldsymbol{u}_k$ と書く.ここで $W_z$ は,$\boldsymbol{w}_z(s_i,t_i)^\top$ を行ベクトルに持つ設計行列である.事前分布は
\[
\boldsymbol{\beta}_k \sim \mathcal{N}(\boldsymbol{b}_0^{(k)}, B_0^{(k)}),
\qquad
\boldsymbol{u}_k \sim \mathcal{N}(\boldsymbol{0}, Q_k^{-1})
\]
とする($Q_k$ は NNGP の精度行列).

点過程の場合と全く同様に,
\[
\theta_k =
\begin{pmatrix}
\boldsymbol{\beta}_k \\
\boldsymbol{u}_k
\end{pmatrix},
\quad
\mu_{0,k} =
\begin{pmatrix}
\boldsymbol{b}_0^{(k)} \\
\boldsymbol{0}
\end{pmatrix},
\quad
R_{0,k} =
\begin{pmatrix}
(B_0^{(k)})^{-1} & 0 \\
0 & Q_k
\end{pmatrix},
\quad
H_z = [\, W_z\ \ I_{n_X} \,],
\]
と置くと,$\theta_k$ に関する(比例)事後密度は
\[
\log p(\theta_k \mid \cdots)
=
-\frac12 (H_z\theta_k - \boldsymbol{z}_k)^\top \Omega_k (H_z\theta_k - \boldsymbol{z}_k)
-\frac12 (\theta_k - \mu_{0,k})^\top R_{0,k}(\theta_k - \mu_{0,k})
+ \text{const},
\]
であり,二乗完成の結果,
\[
\boxed{
\theta_k \mid \Xi, \mathcal{Y}, X
\sim
\mathcal{N}\bigl(\boldsymbol{m}_k, V_k\bigr),
}
\]
\[
V_k^{-1} = H_z^\top \Omega_k H_z + R_{0,k},
\qquad
\boldsymbol{m}_k = V_k \bigl(H_z^\top \Omega_k \boldsymbol{z}_k + R_{0,k}\mu_{0,k}\bigr).
\]

これを $k=1,\dots,K-1$ について繰り返すことで,全てのマーク・カテゴリに対する $(\boldsymbol{\beta}_k, \boldsymbol{u}_k)$ を更新できる.

\subsection{Polya--Gamma 変数 \texorpdfstring{$\xi_{ik}$}{ξ\_ik} の完全条件付き分布}

各 $(i,k)$ について,出てくる形は
\[
\exp\Bigl\{ \tilde{\kappa}_{ik}\eta_{ik} - \frac{\xi_{ik}\eta_{ik}^2}{2}\Bigr\}p(\xi_{ik}),
\]
という形になっているので,点過程部分と同じ議論から,
\[
\boxed{
\xi_{ik} \mid \eta_{ik}, N_i
\sim
\mathrm{PG}\bigl(N_i,\ \eta_{ik}\bigr).
}
\]
ここで $N_i$ は遺跡 $i$ での総出土数である.
