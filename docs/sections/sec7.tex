\section{ギブスサンプリングアルゴリズム(まとめ)}

以上をまとめると,本 joint モデルのギブスサンプリングは概略次のようになる:

\textbf{1.} 初期値
\[
\lambda^{*(0)},\ \boldsymbol{\beta}_{\mathrm{int}}^{(0)},\ \boldsymbol{u}_{\mathrm{int}}^{(0)},
\ \{\boldsymbol{\beta}_k^{(0)},\boldsymbol{u}_k^{(0)}\}_{k=1}^{K-1},
\ \boldsymbol{\omega}^{(0)},\ \Xi^{(0)},\ U^{(0)}
\]
を適当に設定する.

\textbf{2.} 各反復 $\tau = 1,\dots,T$ で:

\textbf{(a) 偽不在の更新}
\[
U^{(\tau)} \sim \mathrm{IPP}\bigl(\lambda^{*(\tau-1)}(1-q^{(\tau-1)})\bigr)
\]
を Poisson thinning によってサンプルする.

\textbf{(b) 点過程側 Polya--Gamma の更新}
\[
\omega_i^{(\tau)}
\sim
\mathrm{PG}\bigl(1,\ \eta_{\mathrm{int},i}^{(\tau-1)}\bigr),
\quad i=1,\dots,n_X+n_U^{(\tau)}.
\]

\textbf{(c) $(\boldsymbol{\beta}_{\mathrm{int}},\boldsymbol{u}_{\mathrm{int}})$ の更新}
\[
\theta_{\mathrm{int}}^{(\tau)}
=
\begin{pmatrix}
\boldsymbol{\beta}_{\mathrm{int}}^{(\tau)} \\
\boldsymbol{u}_{\mathrm{int}}^{(\tau)}
\end{pmatrix}
\sim
\mathcal{N}\bigl(\boldsymbol{m}_{\mathrm{int}}, V_{\mathrm{int}}\bigr),
\]
ただし $\boldsymbol{m}_{\mathrm{int}}, V_{\mathrm{int}}$ は $\boldsymbol{\omega}^{(\tau)}, X, U^{(\tau)}$ から計算する.

\textbf{(d) $\lambda^*$ の更新}
\[
\lambda^{*(\tau)} \sim \mathrm{Ga}\bigl(m_0 + n^{(\tau)},\ r_0 + |\mathcal{D}|\bigr),
\]
ここで $n^{(\tau)} = n_X + n_U^{(\tau)}$.

\textbf{(e) マーク側 Polya--Gamma の更新}

各 $i,k$ について
\[
\xi_{ik}^{(\tau)} \sim \mathrm{PG}\bigl(N_i,\ \eta_{ik}^{(\tau-1)}\bigr).
\]

\textbf{(f) 各カテゴリ $k$ の $(\boldsymbol{\beta}_k,\boldsymbol{u}_k)$ の更新}
\[
\theta_k^{(\tau)}
=
\begin{pmatrix}
\boldsymbol{\beta}_k^{(\tau)} \\
\boldsymbol{u}_k^{(\tau)}
\end{pmatrix}
\sim
\mathcal{N}\bigl(\boldsymbol{m}_k, V_k\bigr),
\]
ただし $\boldsymbol{m}_k, V_k$ は $\Xi^{(\tau)}, \mathcal{Y}, X$ から計算する.
