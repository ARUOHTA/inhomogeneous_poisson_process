\section{時間依存構造を持つ DGP の導入}

\subsection{基本的な考え方}

これまでの定式化では、「時期 $t$ はあくまで共変量の 1 つ」であり、空間ランダム効果 $u_{\mathrm{int}}(s)$ や $u_k(s)$ は「空間だけの GP」として独立に定義していた。

しかし実際には、「早期の分布と中期の分布には連続性がある」「時間が近いほど似た空間パターンをとる」と考えるのが自然である。そこで、以下のような構造を導入する:

\begin{itemize}
\item 時間 $t=1,\dots,T$ ごとに、空間ランダム効果の場 $u_{\mathrm{int}}^{(t)}(s)$, $u_k^{(t)}(s)$ を定義する。
\item これらの場どうしに、時間方向の相関を持たせる(近い時点ほど強く相関する)。
\end{itemize}

このような「時間インデックス付き GP の族」をまとめて 1 つの GP と見る立場を、ここでは DGP と呼ぶことにする。

形式的には、領域 $\mathcal{D}\subset\mathbb{R}^2$ と離散時点 $t\in\{1,\dots,T\}$ を合わせた積空間
\[
\mathcal{D}_\mathrm{st} = \mathcal{D}\times\{1,\dots,T\}
\]
上の GP
\[
u_{\mathrm{int}}(s,t),\quad u_k(s,t)
\]
を定義し、共分散構造で時間依存を表現する。

\subsection{強度側 DGP の具体形}

IPP の強度側で用いていたロジット線形予測子は、時間を明示すると
\[
\eta_{\mathrm{int}}(s,t)
=
\boldsymbol{w}_{\mathrm{int}}(s,t)^\top\boldsymbol{\beta}_{\mathrm{int}}
+
u_{\mathrm{int}}(s,t)
\]
であり、存在確率は
\[
q(s,t)
=
\frac{\exp\{\eta_{\mathrm{int}}(s,t)\}}
{1+\exp\{\eta_{\mathrm{int}}(s,t)\}}.
\]

ここで、$u_{\mathrm{int}}(s,t)$ に対し、空間--時間 GP を
\[
u_{\mathrm{int}}(\cdot,\cdot)
\sim
\mathrm{GP}\bigl(0,\ C_{\mathrm{int}}((s,t),(s',t'))\bigr)
\]
として定義する。もっとも単純で扱いやすいのは「空間と時間の直積(セパラブル)共分散」であり、
\[
C_{\mathrm{int}}((s,t),(s',t'))
=
\sigma_{\mathrm{int}}^2\,
C_{\mathrm{space}}^{\mathrm{int}}(s,s';\phi_{\mathrm{int}})
\,
C_{\mathrm{time}}^{\mathrm{int}}(t,t';\rho_{\mathrm{int}})
\]
のように書く。ここで

\begin{itemize}
\item $C_{\mathrm{space}}^{\mathrm{int}}(s,s';\phi_{\mathrm{int}})$ は空間距離に依存する共分散(例:指数型, Mat\'ern など)
\item $C_{\mathrm{time}}^{\mathrm{int}}(t,t';\rho_{\mathrm{int}}) = \rho_{\mathrm{int}}^{|t-t'|}$ のような AR(1) 型の時間共分散
\item $\sigma_{\mathrm{int}}^2$ は分散スケール
\end{itemize}

とする。

このとき、観測・偽不在の点を合わせた空間--時間位置
\[
(s_1,t_1),\dots,(s_n,t_n)
\]
で評価したベクトル
\[
\boldsymbol{u}_{\mathrm{int}}
=
\bigl(u_{\mathrm{int}}(s_1,t_1),\dots,u_{\mathrm{int}}(s_n,t_n)\bigr)^\top
\]
は
\[
\boldsymbol{u}_{\mathrm{int}}
\sim
\mathcal{N}\bigl(\boldsymbol{0},\ \Sigma_{\mathrm{int}}\bigr)
\]
に従い,$\Sigma_{\mathrm{int}}$ は上記カーネルから構成される共分散行列になる。

セパラブル構造を採用し,時点ごとの空間位置集合が同じ(あるいはグリッド上)であれば,
\[
\Sigma_{\mathrm{int}} = \Sigma_\mathrm{time}^{\mathrm{int}}\ \otimes\ \Sigma_\mathrm{space}^{\mathrm{int}}
\]
という Kronecker 積の形を持つ。ここで

\begin{itemize}
\item $\Sigma_\mathrm{time}^{\mathrm{int}}$ は $T\times T$ の時間共分散行列
\item $\Sigma_\mathrm{space}^{\mathrm{int}}$ は空間位置についての共分散行列
\end{itemize}

である。セパラブル共分散の精度行列は
\[
Q_{\mathrm{int}}
=
\Sigma_{\mathrm{int}}^{-1}
=
(\Sigma_\mathrm{time}^{\mathrm{int}})^{-1} \otimes (\Sigma_\mathrm{space}^{\mathrm{int}})^{-1}
=
Q_\mathrm{time}^{\mathrm{int}}\ \otimes\ Q_\mathrm{space}^{\mathrm{int}}
\]
となる。NNGP を使うときは、空間側の精度行列 $Q_\mathrm{space}^{\mathrm{int}}$ を NNGP による疎な精度行列で近似する。

いずれにせよ、「$\boldsymbol{u}_{\mathrm{int}}$ は多次元正規であり、事前分布は
\[
\boldsymbol{u}_{\mathrm{int}} \sim \mathcal{N}(\boldsymbol{0}, Q_{\mathrm{int}}^{-1})
\]
という形に保たれる」ことが重要である。

\subsection{マーク側 DGP の具体形}

マーク側の線形予測子も,時間を明示すると
\[
\eta_k(s,t)
=
\boldsymbol{w}_z(s,t)^\top\boldsymbol{\beta}_k
+
u_k(s,t),\quad k=1,\dots,K-1,
\]
\[
\pi_k(s,t)
=
\frac{\exp\{\eta_k(s,t)\}}
{\sum_{k'=1}^K\exp\{\eta_{k'}(s,t)\}},
\quad
\eta_K(s,t)\equiv 0
\]
であった。

ここでも同様に,「カテゴリ $k$ ごとに空間--時間 GP を定義」する:
\[
u_k(\cdot,\cdot)
\sim
\mathrm{GP}\bigl(0,\ C_k((s,t),(s',t'))\bigr),
\]
\[
C_k((s,t),(s',t'))
=
\sigma_k^2\,
C_{\mathrm{space}}^{(k)}(s,s';\phi_k)
\,
C_{\mathrm{time}}^{(k)}(t,t';\rho_k),
\]
とする。簡単のため,強度側と同じ形式(セパラブル,時間 AR(1))を用いる。

各時期 $t$ における観測遺跡 $i=1,\dots,n_{X}^{(t)}$ の空間位置 $s_i^{(t)}$ を集め,すべての時期をまとめたインデックスを $i=1,\dots,n_X$ と書くと,
\[
\boldsymbol{u}_k
=
\bigl(u_k(s_1,t_1),\dots,u_k(s_{n_X},t_{n_X})\bigr)^\top
\sim
\mathcal{N}\bigl(\boldsymbol{0}, Q_k^{-1}\bigr),
\]
という形で表現できる。ここで $Q_k$ は NNGP+時間共分散から構成した精度行列である。

このようにして、IPP 強度側の $u_{\mathrm{int}}(s,t)$ と、マーク側の $u_k(s,t)$ の両方に対して、時間依存を持つ DGP 構造を入れたことになる。
