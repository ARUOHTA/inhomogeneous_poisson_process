\documentclass[xelatex, 8pt]{beamer}
\mode<presentation>{\usetheme{Dresden}}
% Boadilla
\usecolortheme[RGB={22, 74, 132}]{structure}
%\usefonttheme{professionalfonts}

\usepackage{xeCJK}
\setCJKmainfont{Noto Serif CJK JP}
%\renewcommand{\familydefault}{\sfdefault}

% IPAexMincho
% Noto Serif CJK JP

\usepackage{tikz}
\usetikzlibrary{intersections, calc, arrows.meta}

%seagullいいね
%
\usepackage{amsmath}
\usepackage{amsthm}
\usepackage{algorithm}
\usepackage{algorithmic}
\usepackage{subcaption}  %図を横に配置してそれぞれにキャプションを追加

% 定理環境で使う言葉を用意
\theoremstyle{plain}
\newtheorem{thm}{Theorem}
\newtheorem*{thm*}{Theorem}

\theoremstyle{definition}
\newtheorem{dfn}{Definition}

% フォントサイズの設定
\setbeamerfont{itemize/enumerate body}{size=\normalsize}
\setbeamerfont{itemize/enumerate subbody}{size=\normalsize}
\setbeamerfont{itemize/enumerate subsubbody}{size=\normalsize}

% リンクに関するセットアップ
\usepackage{url}

\usepackage[dvipdfmx]{color, hyperref}
\usepackage{cite}

% beamerではなぜかこれが必要らしい
%\hypersetup{pdfborder={0 0 1}}
\usepackage{xcolor}
\hypersetup{
	colorlinks=true,
	citecolor=blue,
	linkcolor=red,
	urlcolor=orange,
}

%ページ番号
\setbeamertemplate{footline}[frame number]

%%%%%%%%%%%%%%%%%%%%%%%%%%%%%% Metadata %%%%%%%%%%%%%%%%%%%%%%%%%%%%%%
\hypersetup
{
	%Separate multiple authors by comma
	pdfauthor={},
	pdftitle={Spatial Statistical Models for Obsidian Source Composition: Progress and Challenges},
	pdfsubject={},
	pdfkeywords={},
	colorlinks=false
}

%%%%%%%%%%%%%%%%%%%%%%%%%%%%%% Title related %%%%%%%%%%%%%%%%%%%%%%%%%%%%%%

\title[Contact: Aru Ohta (otaru1214@gmail.com)]{黒曜石産地構成比の空間統計モデル}
\subtitle{Spatial Statistical Models for Obsidian Source Composition}
\date[2025]{2025-10-14}
\author[M2 Aru Ohta]{M2 Aru Ohta}
\institute[Kyoto University]{京都大学情報学研究科}


%%%%%%%%%%%%%%%%%%%%%%%%%%%%% Presentation begins here %%%%%%%%%%%%%%%%%%%%%%%%%



\begin{document}

\frame{\titlepage}

\begin{frame}
{\Large 目次 Contents}
 \tableofcontents
\end{frame}

\section{これまでの振り返り}

\begin{frame}
{\Large 目次}
 \tableofcontents[currentsection]
\end{frame}

\begin{frame}{研究目的}

    \begin{itemize}
        \item 目的:任意の地点における黒曜石の\textbf{産地構成比}を推定する
        \\[2mm]
        \item そのために、
        \\[2mm]
            \begin{itemize}
                \item 黒曜石が出土した遺跡の位置
                \item それぞれの遺跡から出土したそれぞれの産地の黒曜石の出土数
            \end{itemize}
            \\[2mm]
            の2つをモデル化することを考える。
            \\[2mm]
            前者についてのモデルを\textbf{遺跡の存在確率モデル}、後者についてのモデルを\textbf{産地構成比モデル}とよぶことにする。

    \end{itemize}

\end{frame}


\begin{frame}{データの基本情報}
    \begin{itemize}
        \item \textbf{対象領域}: 関東地方(北緯34-37度、東経138-141度)
        \item \textbf{データ規模}:
        \begin{itemize}
            \item 遺跡数: 224箇所
            \item 総出土数: 31,244点
        \end{itemize}
        \vspace{2mm}
        \item \textbf{時期区分}:
        \begin{itemize}
            \item 早期・早々期 (約12,000年前〜7,000年前): 53遺跡
            \item 前期 (約7,000年前〜5,500年前): 61遺跡
            \item 中期 (約5,500年前〜4,500年前): 146遺跡
            \item 後期 (約4,500年前〜3,500年前): 59遺跡
            \item 晩期 (約3,500年前〜2,800年前): 18遺跡
        \end{itemize}
        \vspace{2mm}
        \item \textbf{産地分類}:
        \begin{itemize}
            \item 神津島
            \item 信州(和田峠、男女倉、諏訪、蓼科)
            \item 箱根
            \item 高原山
        \end{itemize}
    \end{itemize}
\end{frame}


\begin{frame}{黒曜石の産地構成比データ}

\begin{figure}
    \centering
    \includegraphics[width=0.7\linewidth]{fig/obsidian_map.png}
    \caption{遺跡と産地の位置}
    \label{fig:placeholder}
\end{figure}

\end{frame}

\begin{frame}{黒曜石の産地構成比データ}

\begin{table}[t]
    \centering
    \scriptsize
    \setlength{\tabcolsep}{2.5pt}
    \caption{黒曜石データにおける独立変数と従属変数の例}
    \begin{tabular}{lrrrrrrrrr|rrrr}
    \hline
    \multicolumn{10}{c|}{独立変数(立地・コスト)} & \multicolumn{4}{c}{従属変数(カウント)} \\
    \hline
    遺跡名 & 経度 & 緯度 & 標高[m] & 傾斜[°] & 信州まで & 神津島まで & 箱根まで & 高原山まで & 河川まで & 信州 & 神津島 & 箱根 & 高原山 \\
    \hline
    上谷津第2 & 140.2516 & 35.6240 & 38.0 & 3.0 & 4089 & 3776 & 2112 & 2674 & 15 & 23 & 29 & 0 & 2 \\
    八幡脇     & 140.2547 & 36.0823 & 24.7 & 0.8 & 3414 & 4403 & 2738 & 1983 & 13 & 6  & 10 & 0 & 0 \\
    陸平貝塚   & 140.3484 & 36.0177 & 23.2 & 3.3 & 3601 & 4401 & 2736 & 2171 & 24 & 0  & 53 & 0 & 0 \\
    後野A     & 140.5578 & 36.3948 & 21.0 & 0.9 & 3627 & 5174 & 3510 & 1935 & 10 & 0  & 0  & 0 & 0 \\
    粟島台     & 140.8391 & 35.7177 & 11.0 & 1.5 & 4525 & 4512 & 2847 & 3095 & 3  & 4  & 70 & 0 & 0 \\
    \multicolumn{1}{c}{$\vdots$} & $\vdots$ & $\vdots$ & $\vdots$ & $\vdots$ & $\vdots$ & $\vdots$ & $\vdots$ & $\vdots$ & $\vdots$ & $\vdots$ & $\vdots$ & $\vdots$ & $\vdots$ \\    \hline
    \end{tabular}
    \par\smallskip
    \begin{minipage}{0.96\textwidth}\footnotesize
    注)距離はコスト距離(単位は解析に依存)
    \end{minipage}

  \end{table}

\end{frame}

\begin{frame}{黒曜石の産地構成比データ}

\begin{figure}
        \centering
        \includegraphics[width=1\linewidth]{fig/独立変数.png}
        \caption{黒曜石データにおける独立変数の例}
        \label{fig:placeholder}
    \end{figure}
\end{frame}




\begin{frame}{モデル化の方針}

    これまで、2つの問題に対して、以下のモデルを実装してきた:

    \vspace{3mm}

    \textbf{1: 遺跡の存在確率モデル(非斉次ポアソン過程)}
    \begin{itemize}
        \item $X \sim \text{IPP}(\lambda)$
        \item $\lambda(s) = \lambda^* \cdot \frac{\exp(\boldsymbol{W}(s)^\top \boldsymbol{\beta})}{1 + \exp(\boldsymbol{W}(s)^\top \boldsymbol{\beta})}$
        \item 説明変数: 標高、傾斜、産地からの距離、河川距離
    \end{itemize}

    \vspace{3mm}

    \textbf{2: 産地構成比モデル(ノンパラメトリック回帰)}
    \begin{itemize}
        \item Nadaraya-Watson推定量(頻度主義)
        \item Kernel Stick-Breaking Process(ベイズ)
        \item Tobler's Hiking Functionによる地形考慮距離
    \end{itemize}
\end{frame}

\begin{frame}{1. 非斉次ポアソン過程モデル}
    非斉次ポアソン過程$X$を用いて遺跡の空間分布をモデル化する。観測領域$\mathcal{D}$上での計数過程$X$が強度$\lambda(s)$を持つ非斉次ポアソン過程に従うとき、任意の可測集合$D \subset \mathcal{D}$に対して
    $$X(D) \sim \text{Poisson}\left(\int_D \lambda(s) ds\right)$$
    が成り立つ。

    遺跡の存在確率を考慮するため、強度関数を以下のように分解する:
    $$\lambda(s) = \lambda^* \cdot q(s)$$
    ここで$\lambda^* > 0$は強度の上限値、$q(s)$は位置$s$での存在確率を表し、
    $$q(s) = \frac{\exp(\boldsymbol{W}(s)^\top \boldsymbol{\beta})}{1 + \exp(\boldsymbol{W}(s)^\top \boldsymbol{\beta})}$$
    として定義される。$\boldsymbol{W}(s)$は位置$s$での説明変数ベクトル(標高、傾斜、産地からの距離など)である。
\end{frame}

\begin{frame}{1. 非斉次ポアソン過程モデル}
    尤度関数に含まれる計算困難な積分項$\int_{\mathcal{D}} q(s) ds$を回避するため、Moreira and Gamerman (2022)\cite{Moreira2022}の潜在変数アプローチを採用する。

    偽不在を表す潜在点過程$U$を導入:
    $$U \sim \text{IPP}(\lambda^*(1-q))$$

    この拡張により、同時尤度は以下のように表現される:
    $$P(X, U \mid \boldsymbol{\beta}, \lambda^*) = \exp \left( - \lambda^* |\mathcal{D}| \right) \cdot \frac{(\lambda^*)^{n}}{n_X!n_U!} \prod_{i=1}^{n} \frac{\left\{\exp\left( \boldsymbol{W}(s_i)^\top \boldsymbol{\beta} \right)\right\}^{y_i}}{1 + \exp\left(\boldsymbol{W}(s_i)^\top \boldsymbol{\beta}\right)}$$

    ここで$n = n_X + n_U$、$y_i$は観測点$(y_i=1)$と潜在点$(y_i=0)$を区別する二値変数である。Polya-Gamma変数を用いることで効率的なギブスサンプリングが可能となり、各パラメータの事後分布は解析的に求まる。
\end{frame}

\begin{frame}{1. 非斉次ポアソン過程モデル: 結果}
    \begin{figure}\centering\includegraphics[width=0.6\textwidth]{fig/trace_site_probability.png}\caption{各パラメータの事後分布とトレースプロット。上から、標高、傾斜角度、神津島からの距離、信州からの距離、箱根からの距離、高原山からの距離、河川からの距離}
    \end{figure}
\end{frame}

\begin{frame}{1. 非斉次ポアソン過程モデル: 結果}
    \begin{figure}\centering\includegraphics[width=0.8\textwidth]{fig/site_probability.png}\caption{領域全体での予測}
    \end{figure}
\end{frame}

\begin{frame}{2-0. 線形回帰モデル(baseline)}

    パラメトリックモデルの最も基本的な例として、線形回帰モデルを実装した。

    \vspace{3mm}
    \textbf{モデル定式化}:

    遺跡$i$における産地$k$の観測カウントを$y_{ik}$とし、多項分布として
    $$\mathbf{y}_i \sim \text{Multinomial}(N_i, \boldsymbol{\pi}(s_i))$$

    $$\pi_k(s) = \frac{\exp(\eta_k(s))}{1 + \sum_{j=1}^{K-1} \exp(\eta_j(s))}$$

    産地構成比に対してAdditive Log-Ratio (ALR)変換を適用し、単純な線形回帰モデルで推定。
    $$\eta_k(s) = \log\left(\frac{\pi_k(s)}{\pi_K(s)}\right) = \beta_{k0} + \beta_{k1} W(s) $$

    ここで$W(s)$は共変量であり、ここでは標高を使う。
\end{frame}

\begin{frame}{2-0. 線形回帰モデル(baseline}
    \begin{figure}
    \centering
    \begin{subfigure}{0.45\textwidth}
        \centering
        \includegraphics[width=\textwidth]{fig/fixed_bayesian_map_2_神津島.png}
        \label{fig:sigma300}
    \end{subfigure}
    \hfill
    \begin{subfigure}{0.45\textwidth}
        \centering
        \includegraphics[width=\textwidth]{fig/fixed_bayesian_map_2_信州.png}
        \label{fig:sigma700}
    \end{subfigure}

    \begin{subfigure}{0.45\textwidth}
        \centering
        \includegraphics[width=\textwidth]{fig/fixed_bayesian_map_2_箱根.png}
        \label{fig:sigma1000}
    \end{subfigure}
        \hfill
    \begin{subfigure}{0.45\textwidth}
        \centering
        \includegraphics[width=\textwidth]{fig/fixed_bayesian_map_2_高原山.png}
        \label{fig:sigma1500}
    \end{subfigure}
\end{figure}
\end{frame}

\begin{frame}{2-1. Nadaraya-Watsonモデル}
    遺跡$i$における産地$k$の黒曜石出土数を$y_{ik}$とし、位置$s_i$での産地構成比$\pi_k(s_i) = y_{ik}/\sum_{k'}y_{ik'}$を観測値とする回帰問題として定式化する。

    任意の位置$s$における産地$k$の構成比$\pi_k(s)$を推定するため、以下のNadaraya-Watson推定量を用いる:

    $$\hat{\pi}_k(s) = \frac{\sum_{i=1}^{n_X} K_h(d(s,s_i)) \cdot y_{ik}}{\sum_{i=1}^{n_X} K_h(d(s,s_i)) \cdot \sum_{k'} y_{ik'}}$$

    ここで$K_h(\cdot)$はバンド幅$h$を持つカーネル関数、$d(s,s')$は2点間の距離である。ここではガウスカーネル
    $$K_h(d) = \frac{1}{h^2} \exp \left(-\frac{d^2}{2h^2}\right)$$
    を採用し、各遺跡からの観測値を距離に応じて重み付け平均することで連続的な産地構成比の空間分布を推定する。
\end{frame}

\begin{frame}{2-1. Nadaraya-Watsonモデル}
    カーネルに使用する距離関数については、従来のユークリッド距離に代わり、地形の起伏を考慮したTobler's Hiking Functionに基づく移動コスト距離を用いる。隣接する2地点間の移動速度(km/h)は勾配$S = \tan\theta$に対して
    $$W = 6e^{-3.5|S+0.05|}$$
    で与えられる。この速度から移動時間を計算し、領域全体を250mメッシュに分割してグラフネットワークを構築する。
    各メッシュ間の移動コストを辺重みとし、全頂点間の最短経路問題として
    $$C_u(v) \leftarrow \min\left(\{C_u(v)\}\cup \{C_u(w) + t_{w\rightarrow v} \mid w \in \mathcal{N}_v\}\right)$$
    の更新式により計算する。海上移動は木造丸木船の速度4km/hを仮定し、沿岸部の地形に応じて上陸コストを設定する。

\end{frame}

\begin{frame}{2-1. Nadaraya-Watsonモデル: 結果}

\begin{figure}\centering\includegraphics[width=0.9\textwidth]{fig/obsidian_ratio_all_NW.png}\caption{横軸が時期(左から順)、縦軸が産地(上から神津島、信州、箱根、高原山)を表す}
\end{figure}

\end{frame}

\begin{frame}{2-1. Nadaraya-Watsonモデル: 結果}
\begin{figure}\centering\includegraphics[width=1\textwidth]{fig/obsidian_ratio_0_神津島.png}\caption{早期・早々期、神津島}
\end{figure}
\end{frame}

\begin{frame}{2-2. Multinomial KSBPモデル}
    Dunson and Park (2008)のKernel Stick-Breaking Process (KSBP)を多項分布データに拡張し、産地構成比の空間変動をノンパラメトリックベイズ的にモデル化する。観測空間$\mathcal{X} \subset \mathbb{R}^d$上で、位置$x$に依存する確率測度$G_x$を以下のStick-breaking表現で構成する:

    $$G_x = \sum_{h=1}^{\infty} \pi_h(x) G_h^*$$

    ここで$\pi_h(x)$は位置$x$での第$h$成分の重み、$G_h^*$は各成分に対応する確率測度である。重み$\pi_h(x)$は空間的局所性を反映するよう以下で定義される:
    $$\pi_h(x) = W_h(x) \prod_{l < h} [1 - W_l(x)]$$
    $$W_h(x) = V_h K(x, \Gamma_h)$$

    $V_h \sim \text{Beta}(1, \lambda)$はstick-breaking比率、$\Gamma_h \sim H$は空間的位置パラメータ、$K(x, \Gamma)$はガウスカーネル$\exp(-\|x-\Gamma\|^2/(2h^2))$である。
\end{frame}

\begin{frame}{2-2. Multinomial KSBPモデル}
    各クラスタ$h$における産地構成比$\theta_h$をディリクレ分布から生成し、混合成分を離散測度$G_h^* = \delta_{\theta_h}$として設定する:
    $$\theta_h \sim \text{Dirichlet}\left(\frac{\gamma_0}{K}, \ldots, \frac{\gamma_0}{K}\right)$$

    位置$s$での産地構成比は無限混合として表現される:
    $$\pi(s) = \sum_{h=1}^{\infty} \pi_h(s) \theta_h$$

    観測データは多項分布に従うとモデル化し、補助変数$z_i$を導入してクラスタ割り当てを明示的に扱う:
    $$z_i \mid s_i \sim \text{Categorical}(\pi_1(s_i), \pi_2(s_i), \ldots)$$
    $$\mathbf{y}_i \mid z_i \sim \text{Multinomial}(N_i, \theta_{z_i})$$

    この階層構造により、空間的に近い遺跡は類似したクラスタに割り当てられ、産地構成比の空間的連続性が自然に表現される。
\end{frame}

\begin{frame}{2-2. Multinomial KSBPモデル}
    各パラメータの完全条件付き事後分布は以下のように導出される。産地構成比パラメータ$\theta_h$については、ディリクレ分布の共役性により
    $$\theta_h \mid \mathbf{z}, \mathbf{y} \sim \text{Dirichlet}\left(\frac{\gamma_0}{K} + S_{h1}, \ldots, \frac{\gamma_0}{K} + S_{hK}\right)$$
    が得られる。ここで$S_{hk} = \sum_{i:z_i=h} y_{ik}$はクラスタ$h$における産地$k$の総出土数である。

    クラスタ割り当て$z_i$の事後分布はカテゴリカル分布となり、未規格化重み
    $$w_{ih} = \pi_h(s_i) \prod_{k=1}^K \theta_{hk}^{y_{ik}}$$
    を用いて$P(z_i = h \mid \text{rest}) = w_{ih}/\sum_g w_{ig}$で与えられる。

    Stick-breaking比率$V_h$については、Walker (2007)のslice samplingを適用し、$V_h \mid \text{rest} \sim \text{Beta}(1+m_h, \lambda+r_h)$の更新式を得る。位置パラメータ$\Gamma_h$は解析的な事後分布が得られないため、ランダムウォークMetropolis-Hastings法を用いる。
\end{frame}

\begin{frame}{2-2. Multinomial KSBPモデル: 結果}

KSBPモデルの早期・早々期の神津島、信州の結果。

\begin{figure}
    \centering
    \begin{subfigure}{0.49\textwidth}
        \centering
        \includegraphics[width=\textwidth]{fig/KSBP_ratio_map_0_信州.png}
        \caption{$\pi_k(s)$の事後平均: 早期・早々期、神津島}
        \label{fig:pointprocess1}
    \end{subfigure}
    \hfill
    \begin{subfigure}{0.49\textwidth}
        \centering
        \includegraphics[width=\textwidth]{fig/KSBP_ratio_map_0_神津島.png}
        \caption{$\pi_k(s)$の事後平均: 早期・早々期、信州}
        \label{fig:pointprocess2}
    \end{subfigure}
    \label{fig:both-predictions}
\end{figure}

\end{frame}

\begin{frame}{評価指標: Aitchison Distance}

    Aitchison Distance\cite{Aitchison1982-yc}は、Compositional data間の距離を測定する標準的な指標。

    $$d_A(\mathbf{x}, \mathbf{y}) = \sqrt{\sum_{i=1}^{D} \left(\ln\frac{x_i}{g(\mathbf{x})} - \ln\frac{y_i}{g(\mathbf{y})}\right)^2}$$

    ここで$g(\mathbf{z}) = \left(\prod_{i=1}^D z_i\right)^{1/D}$は幾何平均。

    \begin{figure}
        \centering
        \includegraphics[width=0.4\linewidth]{fig/aichison_distance.png}
        \caption{太田ら(2006)\cite{ohta2006}より引用。}
        \label{fig:enter-label}
    \end{figure}

\end{frame}
\begin{frame}{モデル比較}

遺跡の産地構成比モデルの3つを、上記の評価指標で評価した結果 (Leave-one-out CV)

\begin{table}
  \centering
  \begin{tabular}{l|cc}
      \hline
      Model & Aitchison Distance & 計算コスト(時間) \\
      \hline
      2-0 線形モデル(baseline) & 11.47 & \textbf{15s} \\
      2-1 Nadaraya-Watson & \textbf{9.34} & 70s \\
      2-2 Multinomial KSBP & 9.85 & 175s \\
      \hline
  \end{tabular}
\end{table}

\begin{itemize}
    \item いまのところ、Nadaraya-Watson推定量が最善。
\end{itemize}

\end{frame}

\begin{frame}{ここまでのまとめと、現在抱えている課題}

    \textbf{Nadaraya-Watson}:
    \begin{itemize}
        \item 推定結果が安定しており、直感的に理解しやすい
        \item 信州と神津島の時期別比較などが顕著に見えて考古学的仮説にも合致
        \item ノンパラメトリックなため特徴量の効果が見えづらい
        \item 今後のモデル拡張の余地があまりない
    \end{itemize}

    \textbf{Multinomial KSBP}:
    \begin{itemize}
        \item ベイズモデルとして不確実性も評価できるのは良い
        \item ハイパーパラメータ探索がまだ不十分なため評価が難しい
        \item ノンパラメトリックなので解釈もしづらい
    \end{itemize}

    \vspace{3mm}

    つまり、今後の課題として

    \begin{itemize}
        \item Nadaraya-Watsonと同等またはそれ以上の推定精度と可視化の質を持ち
        \item パラメータを含み、解釈性が高い
    \end{itemize}

    モデルが求められる。

\end{frame}


\section{提案手法}

\begin{frame}
{\Large 目次}
 \tableofcontents[currentsection]
\end{frame}



\begin{frame}{問題の定式化:空間多項ロジスティック回帰}

まず、

\[
\mathcal{S}\subset\mathbb{R}^2,\quad
\mathcal{X}\subset\mathbb{R}^p,\quad
\Delta^{K-1}:=\{\boldsymbol{\pi}\in[0,1]^K:\ \sum_{k=1}^K \pi_k=1\}.
\]

と定義する。ここで、$\mathcal{S}$は一般に地理空間、$\mathcal{X}$は独立変数の空間、$\Delta^{K-1}$は$K-1$次元の単体である。

\vspace{3mm}

観測データ:$s_i \in \mathcal{S} \quad i = 1, \ldots ,n$ において、

\vspace{2mm}
\begin{itemize}
    \item 独立変数: $x_i\in\mathbb{R}^p$
    \item 従属変数:$K$ 次元の非負整数ベクトル $y_i\in(\mathbb{N}\cup \{0\})^K$
\end{itemize}
\vspace{2mm}

を観測するとする。これらの観測データを元に、回帰関数

\[
\pi:\ \mathcal{S}\times\mathcal{X}\ \to\ \Delta^{K-1},\qquad
(s,x)\ \mapsto\ \pi(s,x).
\]

を推定することを考える。

\end{frame}

\begin{frame}{問題の定式化:空間多項ロジスティック回帰}

この回帰関数の出力$\pi(s_i,x_i)$と観測された従属変数$y_i$をつなぐ観測モデルは、固定された $N_i=\sum_{k=1}^K y_{ik}$に対し
\[
  y_i \mid N_i,\ \pi(s_i,x_i)\ \sim\ \mathrm{Multinomial}\bigl(N_i,\ \pi(s_i,x_i
)\bigr)
\]
と、多項分布に独立に従っている、とするのが自然である。

\vspace{3mm}

→\textbf{空間的にパラメータが変化する多項ロジスティック回帰}

\vspace{3mm}

近接する地点では $\pi$ が連続的に変化するような仮定のもとで、$\pi$の推定手法を考える

\end{frame}

\begin{frame}{提案手法:方針}

方針:空間変化するパラメータを持つ、ベイズ型の回帰モデルを考える。

さらに、空間効果を持たせるため、空間変化係数として、回帰係数の事前分布にGaussian Process(GP)を用いる。

\end{frame}

\begin{frame}{提案手法:Multinomial Logit Model with Spatially Varying Coefficient}
    \textbf{空間変化係数}

    \begin{itemize}
      \item 各産地カテゴリ $k = 1,\ldots,K-1$ ごとに、空間的に変化する係数ベクトル
        \[
          \boldsymbol{\beta}_k(s)
          = \bigl(\beta_{0k}(s),\,\beta_{1k}(s),\,\ldots,\,\beta_{pk}(s)\bigr)^\top
        \]
        を導入する($p$ :共変量の数)。
      \item これと共変量ベクトル
        \[
          \mathbf{W}(s,x) = \bigl(1,\,W_1(s,x),\,\ldots,\,W_p(s,x)\bigr)^\top
        \]
        を用いて、線形予測子を
        \[
          \eta_k(s,x) = \mathbf{W}(s,x)^\top \boldsymbol{\beta}_k(s)
                       = \sum_{j=0}^{p} W_j(s,x)\,\beta_{jk}(s)
        \]
        と定義する。ここで $W_0(s,x) \equiv 1$ は切片項。
      \item $\boldsymbol{\beta}_k(s)$ が地点 $s$ に依存して変化することで、
            近接する地点では似た線形予測子が得られる
    \end{itemize}
  \end{frame}

\begin{frame}{提案手法:Multinomial Logit Model with Spatially Varying Coefficient}
    \textbf{産地構成比のモデル}

        \begin{itemize}
        \item 線形予測子 $\eta_k(s,x)$ を softmax に通し、構成比
        \[
          \boldsymbol{\pi}(s,x)
          = \bigl(\pi_1(s,x),\,\ldots,\,\pi_K(s,x)\bigr)^\top
        \]
        を得る。
        \item 基準カテゴリを $K$ 番目に取ると、
        \[
          \pi_k(s,x)
            = \frac{\exp\{\eta_k(s,x)\}}
                   {1 + \sum_{\ell=1}^{K-1} \exp\{\eta_\ell(s,x)\}},
          \qquad k = 1,\ldots,K-1,
        \]
        \[
          \pi_K(s,x)
            = \frac{1}
                   {1 + \sum_{\ell=1}^{K-1} \exp\{\eta_\ell(s,x)\}}.
        \]
    \end{itemize}
\end{frame}

\begin{frame}{提案手法:Multinomial Logit Model with Spatially Varying Coefficient}
    \textbf{尤度関数}

    \begin{itemize}
      \item 各観測地点 $s_i$ で、総出土数 $N_i = \sum_{k=1}^K y_{ik}$ は既知とする。
      \item 構成比 $\boldsymbol{\pi}(s_i,x_i)$ が与えられたとき、
            観測されたカウント $\mathbf{y}_i = (y_{i1},\ldots,y_{iK})^\top$ は
        \[
          \mathbf{y}_i \mid N_i,\ \boldsymbol{\pi}(s_i,x_i)
          \sim \operatorname{Multinomial}\!\bigl(N_i,\ \boldsymbol{\pi}(s_i,x_i)\bigr)
        \]
        に独立に従うと仮定する。
      \item 多項分布の確率質量関数は
        \[
          p(\mathbf{y}_i \mid N_i, \boldsymbol{\pi}(s_i,x_i))
          = \frac{N_i!}{\prod_{k=1}^K y_{ik}!}
            \prod_{k=1}^{K} \pi_k(s_i,x_i)^{\,y_{ik}}
        \]
        である。
    \end{itemize}
  \end{frame}

  \begin{frame}{提案手法:Multinomial Logit Model with Spatially Varying Coefficient}
    \textbf{尤度関数を書き下す}

    \begin{itemize}
      \item 線形予測子は $\eta_k(s_i,x_i) = \mathbf{W}(s_i,x_i)^\top \boldsymbol{\beta}_k(s_i)$。
      \item 尤度を書き下す:
    \end{itemize}
    \[
    \begin{aligned}
    p(\mathbf{y}_i \mid N_i,\ \{\boldsymbol{\beta}_k\})
      &= \frac{N_i!}{\prod_{k=1}^K y_{ik}!}
         \prod_{k=1}^{K-1}
         \left(
           \frac{\exp\{\mathbf{W}(s_i,x_i)^\top \boldsymbol{\beta}_k(s_i)\}}
                {1 + \sum_{\ell=1}^{K-1} \exp\{\mathbf{W}(s_i,x_i)^\top \boldsymbol{\beta}_\ell(s_i)\}}
         \right)^{y_{ik}} \\[4pt]
      &\qquad\times
         \left(
           \frac{1}
                {1 + \sum_{\ell=1}^{K-1} \exp\{\mathbf{W}(s_i,x_i)^\top \boldsymbol{\beta}_\ell(s_i)\}}
         \right)^{y_{iK}} \\[6pt]
      &= \frac{N_i!}{\prod_{k=1}^K y_{ik}!}\;
         \exp\!\Biggl\{
           \sum_{k=1}^{K-1}
             y_{ik}\,\mathbf{W}(s_i,x_i)^\top \boldsymbol{\beta}_k(s_i)
         \Biggr\} \\[2pt]
      &\qquad\times
         \left(
           1 + \sum_{\ell=1}^{K-1}
                 \exp\{\mathbf{W}(s_i,x_i)^\top \boldsymbol{\beta}_\ell(s_i)\}
         \right)^{-N_i}.
    \end{aligned}
    \]
    \begin{itemize}
      \item このモデルに対して、事前分布を導入し、その後事後分布を導出する。
    \end{itemize}
  \end{frame}


  \begin{frame}{$\boldsymbol{\beta}(s)$の事前分布:Gaussian Process}

    \begin{itemize}
      \item 各カテゴリ $k$・共変量成分 $j$ の組に対し、
            回帰係数 $\beta_{jk}(s)$ はガウス過程に従うとする:
        \[
          \beta_{jk}(s) \sim \mathcal{GP}\bigl(0,\ C_\theta(s,s')\bigr).
        \]
      \item カーネル $C_\theta$ としてRBF(squared exponential)を採用:
        \[
          C_\theta(s,s')
          = \sigma^2 \exp\!\left(-\frac{\|s - s'\|^2}{2\ell^2}\right),
        \]
        ここで $\sigma^2$、$\ell$ はハイパーパラメータ。
    \end{itemize}
  \end{frame}

  \begin{frame}{$\boldsymbol{\beta}(s)$の事前分布:Gaussian Process}
    \textbf{有限の観測点 $S = \{s_1,\ldots,s_n\}$ 上でのGP}

    \begin{itemize}
      \item 回帰係数ベクトル
        \(
          \boldsymbol{\beta}_{jk,S}
          = \bigl(\beta_{jk}(s_1),\ldots,\beta_{jk}(s_n)\bigr)^\top
        \)
        は多変量正規分布に従う:
        \[
          \boldsymbol{\beta}_{jk,S}
            \sim \mathcal{N}\bigl(\mathbf{0},\ K_\theta\bigr),
          \qquad (K_\theta)_{ab} = C_\theta(s_a,s_b).
        \]
      \item また、観測点外 $u \notin S$ に対しても、
        \[
          \begin{pmatrix}
            \boldsymbol{\beta}_{jk,S} \\
            \beta_{jk}(u)
          \end{pmatrix}
          \sim \mathcal{N}\!\left(
            \mathbf{0},\
            \begin{pmatrix}
              K_\theta & k_\theta(u) \\
              k_\theta(u)^\top & C_\theta(u,u)
            \end{pmatrix}
          \right),
        \]
        となるため、ここから正規分布の条件付き分布の公式を使って予測が可能。
      \item ただ、この完全共分散構造を直接扱うのは計算的に重く、逆行列計算に$O(n^3)$の計算量が必要。
    \end{itemize}
  \end{frame}


\begin{frame}{Nearest-Neighbor Gaussian Processによる近似}
    \begin{itemize}
      \item 観測点の集合 $S = \{s_1,\ldots,s_n\}$ に対し、ガウス過程の完全共分散を扱うと
            $O(n^3)$ の計算が必要であり、現実的でない。
      \item これに対し、Nearest-Neighbor Gaussian Process (NNGP)\cite{Datta2016} では、
            各点 $s_i$ の近傍 $M$ 個だけを
            近傍集合 $N_i$ として参照する。
      \item これにより、条件付き分布
        \[
          p\bigl(\beta(s_i) \mid \beta(S_{N_i})\bigr)
          \approx \mathcal{N}\bigl(a_i^\top \beta(S_{N_i}),\ d_i\bigr)
        \]
        を用いて、ガウス過程の空間的な依存構造を局所化できる。
    \end{itemize}
\end{frame}

\begin{frame}{Nearest-Neighbor Gaussian Processによる近似}

\begin{figure}
    \centering
    \includegraphics[width=0.4\linewidth]{fig/NNGP_demo.png}
    \caption{NNGPの例。\cite{Jones2020NNGP}より引用。}
    \label{fig:placeholder}
\end{figure}

\end{frame}
\begin{frame}{Nearest-Neighbor Gaussian Processによる近似}

ここで $S_{N_i} := \{s_j : j \in N_i\}$、$M$ は固定の近傍数。
$N_i \subset \{1,\ldots,i-1\},\ |N_i| \le M$として、近似を行う:

   \[
  \begin{aligned}
  p(\beta(s_1),\ldots,\beta(s_n))
   &= \prod_{i=1}^n
        p\bigl(\beta(s_i)\mid\beta(s_{<i})\bigr), \\
  p\bigl(\beta(s_i)\mid\beta(s_{<i})\bigr)
   &= \mathcal{N}\Bigl(
        C_\theta(s_i,S_{<i}) C_\theta(S_{<i},S_{<i})^{-1} \beta(S_{<i}),
        \ \sigma_i^2
      \Bigr) \\
   &\approx
      \mathcal{N}\Bigl(
        C_\theta(s_i,S_{N_i}) C_\theta(S_{N_i},S_{N_i})^{-1} \beta(S_{N_i}),
        \ d_i
      \Bigr).
  \end{aligned}
  \]
  \[
  a_i = C_\theta(s_i,S_{N_i}) C_\theta(S_{N_i},S_{N_i})^{-1},\quad
  d_i = C_\theta(s_i,s_i) - C_\theta(s_i,S_{N_i}) C_\theta(S_{N_i},S_{N_i})^{-1} C_\theta(S_{N_i},s_i).
  \]

  \end{frame}

  \begin{frame}{NNGPの同時事前分布の導出}

    一点ずつの条件付き事後が一変量正規として処理できる
    \[
    \beta(s_i)\mid\beta(S_{N_i})\sim\mathcal{N}\bigl(a_i^\top\beta(S_{N_i}),\,d_i\bigr),
    \quad
    a_i=C_\theta(s_i,S_{N_i})\,C_\theta(S_{N_i},S_{N_i})^{-1}.
    \]

    これを分解して表現すると
    \[
    \beta(s_i)
      = a_i^\top\beta(S_{N_i}) + \varepsilon_i,
      \qquad \varepsilon_i \sim \mathcal{N}(0,\,d_i).
    \]

    となる。この式を出発点として、全ての点における同時事前分布を導出したい。

  \end{frame}

  \begin{frame}{NNGPの同時事前分布の導出}
    \textbf{全点の事前分布を同時に並べる}

    \[
    \begin{aligned}
      \beta(s_1) &= \varepsilon_1,\\
      \beta(s_2) &= a_{2,1}\beta(s_1) + \varepsilon_2,\\
      &\;\vdots\\
      \beta(s_n) &= \sum_{j\in N_n} a_{n,j}\,\beta(s_j) + \varepsilon_n,
    \end{aligned}
    \]
    これをベクトル $\boldsymbol{\beta}(S) = (\beta(s_1),\ldots,\beta(s_n))^\top$ と行列 $A$ を用いて
    \[
      \boldsymbol{\beta}(S) = A\,\boldsymbol{\beta}(S) + \boldsymbol{\varepsilon}
    \]
    と書ける。$A$ は厳密下三角で、$A_{ij}=a_{i,j}$ if $j\in N_i$, otherwise $0$。
    残差ベクトル $\boldsymbol{\varepsilon}$ は対角共分散:
    \[
      \boldsymbol{\varepsilon} \sim \mathcal{N}(\mathbf{0},\,D),
      \qquad D=\operatorname{diag}(d_1,\ldots,d_n).
    \]
  \end{frame}

  \begin{frame}{NNGPの同時事前分布の導出}

    $\boldsymbol{\beta}(S)$の分布を求めたい

  \[
  \begin{aligned}
  \boldsymbol{\beta}(S)
  &= A\,\boldsymbol{\beta}(S) + \boldsymbol{\varepsilon}, \qquad
  \boldsymbol{\varepsilon} \sim \mathcal{N}(\mathbf{0}, D), \\
  (I - A)\,\boldsymbol{\beta}(S)
  &= \boldsymbol{\varepsilon}
  \;\Rightarrow\;
  \boldsymbol{\beta}(S) = (I - A)^{-1} \boldsymbol{\varepsilon}, \\
  \widetilde{C}
  &= \operatorname{Cov}\bigl(\boldsymbol{\beta}(S)\bigr)
   = \operatorname{Cov}\Bigl((I - A)^{-1} \boldsymbol{\varepsilon}\Bigr) \\
  &= (I - A)^{-1} \operatorname{Cov}(\boldsymbol{\varepsilon}) (I - A)^{-\top} \\
  &= (I - A)^{-1} D (I - A)^{-\top}.
  \end{aligned}
  \]

  \smallskip
  \(\Rightarrow\) 近似しない場合の共分散 \(C_\theta\) をNNGP近似した場合の共分散が \(\widetilde{C}\)。
  \end{frame}

  \begin{frame}{NNGPの同時事前分布の導出}
  独立な正規分布になることがわかる:

  \[
  \begin{aligned}
  p\bigl(\boldsymbol{\beta}(S)\bigr)
  &= (2\pi)^{-\tfrac{n}{2}} \, |\widetilde{C}|^{-\tfrac{1}{2}}
     \exp\!\Bigl( -\tfrac{1}{2}
       \boldsymbol{\beta}(S)^\top \widetilde{C}^{-1} \boldsymbol{\beta}(S)
     \Bigr), \\
  |\widetilde{C}|
  &= \bigl|(I-A)^{-1} D (I-A)^{-\top}\bigr|
   = |I-A|^{-2}\,|D|, \\
  \widetilde{C}^{-1}
  &= (I - A)^\top D^{-1} (I - A), \\
  \boldsymbol{\beta}(S)^\top \widetilde{C}^{-1} \boldsymbol{\beta}(S)
  &= \bigl((I - A)\boldsymbol{\beta}(S)\bigr)^\top
     D^{-1}
     \bigl((I - A)\boldsymbol{\beta}(S)\bigr) \\
  &= \sum_{i=1}^{n}
     \frac{\bigl(\beta(s_i) - a_i^\top \beta(S_{N_i})\bigr)^2}{d_i}, \\
  \log p\bigl(\boldsymbol{\beta}(S)\bigr)
  &= -\tfrac{1}{2}
     \sum_{i=1}^{n}
       \left[
         \frac{\bigl(\beta(s_i) - a_i^\top \beta(S_{N_i})\bigr)^2}{d_i}
         + \log d_i
       \right]
     - \tfrac{n}{2}\log(2\pi).
  \end{aligned}
  \]

  \smallskip
  \(\Rightarrow\) 各 \(i\) で独立な正規分布。
  \end{frame}

  \begin{frame}{NNGPにおける予測分布}
    観測点外での条件付き分布(予測分布)も同じように正規分布になる。

    別集合 $\mathcal{G} = \{u_1,\ldots,u_m\}$ に対して同様の近似を行うと、
    各 $u \in \mathcal{G}$ について、観測集合 $S$ を条件に
    \[
      \beta(u) \mid \beta(S)
        \approx \mathcal{N}\bigl(a_*(u)\,\beta(S),\ d_*(u)\bigr),
    \]
    ここで
    \[
    \begin{aligned}
      a_*(u) &= C_\theta(u, S_{M(u)})\;
               C_\theta(S_{M(u)}, S_{M(u)})^{-1},\\
      d_*(u) &= C_\theta(u,u)
               - C_\theta(u, S_{M(u)})\;
                 C_\theta(S_{M(u)}, S_{M(u)})^{-1}
                 C_\theta(S_{M(u)}, u),
    \end{aligned}
    \]
    $S_{M(u)}$ は $u$ に最も近い観測点の集合(サイズ $M$)。

    これによって外部点におけるガウス過程の期待値・分散を
    条件付きで計算できる。
  \end{frame}

  \begin{frame}{計算量と近似の精度}
    以上のNNGP近似を行うことで、$O(n^3)$から

    \[
       O(n M^3)
    \]
    と大幅に削減される($M$ は近傍数)。

    \begin{itemize}
      \item $M$ を小さく保てば $O(n)$ に近いスケーリングが得られる。
      \item 典型的には $M=10$--$30$ 程度で十分な精度が得られると報告されている。
      \item 今回のNNGP近似(Vecchia 近似という)は $\widetilde{C}$ が正定値になるよう構成され、
            元のガウス過程の分布と KL ダイバージェンスを最小化する
            という意味で最適な局所近似を与えるとされている。
    \end{itemize}
  \end{frame}

\begin{frame}{モデルの尤度関数}

ここから、事前分布と尤度を組み合わせて事後推論を導出していく:
  \[
  \begin{aligned}
  p(\mathbf{y}_i \mid N_i, \{\boldsymbol{\beta}_k\})
  &= \frac{N_i!}{\prod_{k=1}^K y_{ik}!}
     \prod_{k=1}^{K-1}
       \left(
         \frac{\exp\{\mathbf{W}_i^\top \boldsymbol{\beta}_k\}}
              {1 + \sum_{\ell=1}^{K-1} \exp\{\mathbf{W}_i^\top \boldsymbol{\beta}_\ell\}}
       \right)^{y_{ik}} \\
  &\qquad\times
     \left(
       \frac{1}{1 + \sum_{\ell=1}^{K-1} \exp\{\mathbf{W}_i^\top \boldsymbol{\beta}_\ell\}}
     \right)^{y_{iK}} \\
  &= c_i
     \prod_{k=1}^{K-1}
       \exp\{y_{ik}\,\mathbf{W}_i^\top \boldsymbol{\beta}_k\}
       \left(1 + \sum_{\ell=1}^{K-1} \exp\{\mathbf{W}_i^\top \boldsymbol{\beta}_\ell\}\right)^{-N_i}.
  \end{aligned}
  \]

  ただし、$c_i = \frac{N_i!}{\prod_{k=1}^K y_{ik}!}$。
  \end{frame}

  \begin{frame}{Pólya--Gamma Data Augmentation}

  Theorem.1 of Polson et.al (2013)\cite{Polson2013}
    \[
  \frac{(e^\psi)^a}{(1+e^\psi)^b}
  = 2^{-b} e^{\kappa \psi}
    \int_0^\infty \exp\!\left(-\frac{\omega \psi^2}{2}\right)
                   p_{\mathrm{PG}}(\omega \mid b,0)\,d\omega,
  \qquad \kappa = a-\frac{b}{2}.
  \]
  \[
  \omega \sim \mathrm{PG}(b,0),\quad b>0.
  \]

    これを適用すると、補助変数$\boldsymbol{\omega}_i$の元で尤度は正規分布になる:

  \[
  \begin{aligned}
  p(\mathbf{y}_i \mid N_i, \{\boldsymbol{\beta}_k\})
  &= c_i
     \prod_{k=1}^{K-1}
       2^{-N_i}
       \exp\left\{\left(y_{ik}-\tfrac{N_i}{2}\right)\mathbf{W}_i^\top \boldsymbol{\beta}_k\right\} \\
  &\qquad\times
       \int_0^\infty \exp\left\{-\tfrac{1}{2}\omega_{ik}(\mathbf{W}_i^\top \boldsymbol{\beta}_k)^2\right\}
                     p_{\mathrm{PG}}(\omega_{ik}\mid N_i,0)\,d\omega_{ik} \\
  &= c_i
     \prod_{k=1}^{K-1}
       \mathbb{E}_{\omega_{ik}\sim\mathrm{PG}(N_i,0)}
         \left[
           \exp\left\{
             \left(y_{ik}-\tfrac{N_i}{2}\right)\mathbf{W}_i^\top \boldsymbol{\beta}_k
             -\tfrac{1}{2}\omega_{ik}(\mathbf{W}_i^\top \boldsymbol{\beta}_k)^2
           \right\}
         \right].
  \end{aligned}
  \]
  \[
  p(\mathbf{y}_i \mid N_i, \{\boldsymbol{\beta}_k\}, \boldsymbol{\omega}_i)
  \propto
  \prod_{k=1}^{K-1}
  \exp\left\{
  -\tfrac{\omega_{ik}}{2}
  \left(
  \mathbf{W}_i^\top \boldsymbol{\beta}_k
  -\frac{y_{ik}-\frac{N_i}{2}}{\omega_{ik}}
  \right)^2
  \right\}.
  \]
  \end{frame}

  \begin{frame}{モデルの尤度関数}

  行列表現でまとめて書くと、多変量正規分布になる:
  \[
  p(\mathbf{y} \mid \{\boldsymbol{\beta}_k\}, \boldsymbol{\omega})
  \propto
  \prod_{k=1}^{K-1}
  \exp\left\{
  -\tfrac{1}{2}
  (\boldsymbol{z}_k - W \boldsymbol{\beta}_k)^\top
  \Omega_k
  (\boldsymbol{z}_k - W \boldsymbol{\beta}_k)
  \right\}.
  \]
  \[
  \Omega_k = \operatorname{diag}(\omega_{1k},\ldots,\omega_{nk}),
  \quad
  \boldsymbol{z}_k =
  \left(
  \frac{y_{1k}-\frac{N_1}{2}}{\omega_{1k}},
  \ldots,
  \frac{y_{nk}-\frac{N_n}{2}}{\omega_{nk}}
  \right)^\top,
  \quad
  W =
  \begin{pmatrix}
  \mathbf{W}_1^\top \\
  \vdots \\
  \mathbf{W}_n^\top
  \end{pmatrix}.
  \]

  \vspace{3mm}

  これと、Gaussian Process Priorを組み合わせて、パラメータの完全条件付き分布を導出していく:
  \end{frame}


 \begin{frame}{$\boldsymbol{\beta}$の完全条件付き分布の導出}
  \[
  \begin{aligned}
  &\log p(\boldsymbol{\beta}_k \mid \mathbf{y}, \boldsymbol{\omega})\\
  &\propto -\tfrac{1}{2}
     (\boldsymbol{z}_k - W \boldsymbol{\beta}_k)^\top
     \Omega_k
     (\boldsymbol{z}_k - W \boldsymbol{\beta}_k)
     - \tfrac{1}{2}
     \sum_{j=0}^{p}
     \sum_{i=1}^{n}
       \frac{
         \bigl(\beta_{jk}(s_i) - a_i^\top \boldsymbol{\beta}_{jk}(S_{N_i})\bigr)^2
       }{d_i} \\
  &= -\tfrac{1}{2}
     \sum_{i=1}^{n}
     \sum_{k=1}^{K-1}
       \omega_{ik}
       \Bigl(
         \mathbf{W}_i^\top \boldsymbol{\beta}_k
         - \tfrac{y_{ik} - \frac{N_i}{2}}{\omega_{ik}}
       \Bigr)^2
     - \tfrac{1}{2}
     \sum_{j=0}^{p}
     \sum_{i=1}^{n}
       \frac{
         \bigl(\beta_{jk}(s_i) - a_i^\top \boldsymbol{\beta}_{jk}(S_{N_i})\bigr)^2
       }{d_i}\\
  &= -\tfrac{1}{2}
     \sum_{i=1}^{n}
     \sum_{j=0}^{p}
       \left[
         \omega_{ik} w_{ij}^2\,\beta_{jk}(s_i)^2
         -2\omega_{ik} w_{ij}\,\beta_{jk}(s_i)\bigl(\tfrac{y_{ik} - \frac{N_i}{2}}{\omega_{ik}} - \eta_k^{(-j)}(s_i)\bigr)
       \right] \\
  &\quad -\tfrac{1}{2}
     \sum_{i=1}^{n}
     \sum_{j=0}^{p}
       \frac{
         \bigl(\beta_{jk}(s_i) - a_i^\top \boldsymbol{\beta}_{jk}(S_{N_i})\bigr)^2
       }{d_i}\\
  &\propto -\tfrac{1}{2}
     \sum_{i=1}^{n}
     \sum_{j=0}^{p}
       \left[
         \bigl(\omega_{ik} w_{ij}^2 + d_i^{-1}\bigr)\,\beta_{jk}(s_i)^2
         - 2\,\beta_{jk}(s_i)\,
           \bigl( \omega_{ik} w_{ij}(\eta_k^{(-j)}(s_i) - C_{ik})
                 + \kappa_{ik} w_{ij}
                 + d_i^{-1} \mu_{ik}^{\text{prior}}
           \bigr)
       \right]
  \end{aligned}
  \]
  \[
  \eta_k^{(-j)}(s_i) = \eta_k(s_i) - w_{ij}\beta_{jk}(s_i),\quad
  C_{ik} = \log\!\Bigl(1 + \sum_{\ell \neq k} \exp\{\eta_\ell(s_i)\}\Bigr),
  \]
  \[
  \kappa_{ik} = y_{ik} - \tfrac{N_i}{2},\quad
  \mu_{ik}^{\text{prior}} = a_i^\top \boldsymbol{\beta}_{jk}(S_{N_i}).
  \]
  \end{frame}

  \begin{frame}{$\boldsymbol{\beta}$の完全条件付き分布の導出}
  $i, j, k$全てについて独立に正規分布となる:
  \[
  \begin{aligned}
  \beta_{jk}(s_i) \mid \text{rest}
  &\sim \mathcal{N}\!\left(
      m_{ikj},\ v_{ikj}
    \right), \\
  v_{ikj}
  &= \frac{1}{
         \omega_{ik} w_{ij}^2 + d_i^{-1}
       }, \\
  m_{ikj}
  &= v_{ikj}
     \left[
       \omega_{ik} w_{ij}\bigl(\eta_k^{(-j)}(s_i) - C_{ik}\bigr)
       + \kappa_{ik} w_{ij}
       + d_i^{-1} \mu_{ik}^{\text{prior}}
     \right].
  \end{aligned}
  \]
  \[
  \eta_k^{(-j)}(s_i) = \eta_k(s_i) - w_{ij}\beta_{jk}(s_i),\quad
  C_{ik} = \log\!\Bigl(1 + \sum_{\ell \neq k} \exp\{\eta_\ell(s_i)\}\Bigr),
  \]
  \[
  \kappa_{ik} = y_{ik} - \tfrac{N_i}{2},\quad
  \mu_{ik}^{\text{prior}} = a_i^\top \boldsymbol{\beta}_{jk}(S_{N_i}).
  \]

  \end{frame}

\begin{frame}{\(\boldsymbol{\omega}\) の完全条件付き分布の導出}
  \[
  \begin{aligned}
  p(\omega_{ik} \mid \boldsymbol{\beta}, \mathbf{y}, \mathbf{W})
  &\propto
  p(\mathbf{y}_i \mid \boldsymbol{\beta}, \omega_{ik})
  \cdot
  p(\omega_{ik} \mid N_i, 0) \\
  &\propto
  \exp\left\{
    \left(y_{ik} - \frac{N_i}{2}\right)\mathbf{W}_i^\top \boldsymbol{\beta}_k
    - \frac{\omega_{ik}}{2}\left(\mathbf{W}_i^\top \boldsymbol{\beta}_k\right)^2
  \right\}
  \cdot
  p(\omega_{ik} \mid N_i, 0) \\
  &\propto
  \exp\left\{
    - \frac{\omega_{ik}}{2}\left(\mathbf{W}_i^\top \boldsymbol{\beta}_k\right)^2
  \right\}
  \cdot
  p(\omega_{ik} \mid N_i, 0).
  \end{aligned}
  \]

  Polson et al. (2013) の Theorem 1 によれば,
  \[
  p(\omega \mid b, \psi)
  \propto
  \exp\left(-\frac{\omega}{2}\psi^{2}\right)
  \cdot
  p(\omega \mid b, 0)
  \;\Rightarrow\;
  \omega \mid \psi \sim \mathrm{PG}(b, \psi).
  \]

  よって,
  \[
  \omega_{ik} \mid \boldsymbol{\beta}, \mathbf{y}, \mathbf{W}
  \sim
  \mathrm{PG}\!\left(N_i,\; \mathbf{W}_i^\top \boldsymbol{\beta}_k\right).
  \]

  となり、以上でギブスサンプラーを導出できた。
  \end{frame}





\section{結果・考察}

\begin{frame}{実験}

提案手法に対して、以下の設定で実験を行った:

\begin{itemize}
    \item 共変量:全ての共変量
    \item カーネル:RBFカーネル(全ての共変量で共通のハイパラを採用)
    \item NNGP近傍数:30
    \item MCMC反復数: 200

\end{itemize}

\end{frame}

\begin{frame}{結果}

\begin{figure}
    \centering
    \includegraphics[width=0.75\linewidth]{fig/nngp_trace.png}
    \caption{$\beta$のトレースプロット。ランダムに選んだ20個を描画。}
    \label{fig:placeholder}
\end{figure}

\end{frame}

\begin{frame}{結果}

\begin{figure}
    \centering
    \includegraphics[width=0.9\linewidth]{fig/obsidian_nngp_multinomial/nngp_multinomial_origin1_period1.png}
    \caption{前期、信州の推定結果。}
    \label{fig:placeholder}
\end{figure}

\end{frame}

\begin{frame}{結果}

前期、中期、後期の神津島と信州の推定結果。
    \begin{figure}
    \centering
    \begin{subfigure}{0.30\textwidth}
        \centering
        \includegraphics[width=\textwidth]{fig/obsidian_nngp_multinomial/nngp_multinomial_origin0_period1.png}
        \label{fig:sigma300}
    \end{subfigure}
    \hfill
    \begin{subfigure}{0.30\textwidth}
        \centering
        \includegraphics[width=\textwidth]{fig/obsidian_nngp_multinomial/nngp_multinomial_origin0_period2.png}
        \label{fig:sigma700}
    \end{subfigure}
    \hfill
    \begin{subfigure}{0.30\textwidth}
        \centering
        \includegraphics[width=\textwidth]{fig/obsidian_nngp_multinomial/nngp_multinomial_origin0_period3.png}
        \label{fig:sigma700}
    \end{subfigure}

    \begin{subfigure}{0.30\textwidth}
        \centering
        \includegraphics[width=\textwidth]{fig/obsidian_nngp_multinomial/nngp_multinomial_origin1_period1.png}
        \label{fig:sigma1000}
    \end{subfigure}
        \hfill
    \begin{subfigure}{0.30\textwidth}
        \centering
        \includegraphics[width=\textwidth]{fig/obsidian_nngp_multinomial/nngp_multinomial_origin1_period2.png}
        \label{fig:sigma1500}
    \end{subfigure}
    \hfill
    \begin{subfigure}{0.30\textwidth}
        \centering
        \includegraphics[width=\textwidth]{fig/obsidian_nngp_multinomial/nngp_multinomial_origin1_period3.png}
        \label{fig:sigma700}
    \end{subfigure}
\end{figure}
\end{frame}

\begin{frame}{考察}

\begin{itemize}
    \item 観測データの比率に基づいて、なめらかに空間的な補完ができているように見える。
    \item MCMCも安定して動き、収束速度も早い。
    \item 課題はモデル構造やカーネルの設計によって事前知識をうまく組み込む(=正則化)ことだが、その元となるモデルとして有用である可能性

    \vspace{2mm}
    \item 特に、GPの平均が0であること、そしてGPによる空間効果の項のみを考えていることにより、大域的な傾向を掴めていない:
\end{itemize}

    \begin{tikzpicture}[>=stealth,scale=1.3]
      % 軸
      \draw[->,thick] (-0.2,0) -- (6.5,0) node[below right]{産地からの距離};
      \draw[->,thick] (0,-0.2) -- (0,2.2) node[left]{比率};

      % 0.25の補助線
      \draw[dashed,gray] (0,0.5) -- (6,0.5);
      \node[left] at (0,0.5){0.25};

      % 曲線(なめらかな補間)
      \draw[thick,decorate,decoration={snake,amplitude=0.2,segment length=6pt}]
        plot[smooth] coordinates {
          (0,0.5)      % 左端:補助線に収束
          (0.6,0.6)
          (1.0,0.8)    % 観測点
          (1.4,1.0)
          (1.8,1.2)    % 観測点
          (2.1,1.05)
          (2.4,0.92)
          (2.8,0.8)    % 観測点
          (3.2,0.62)
          (3.5,0.55)
          (3.8,0.5)    % 観測点
          (4.2,0.46)
          (4.6,0.48)
          (5.2,0.52)
          (5.6,0.50)
          (6.0,0.5)    % 右端:補助線に収束
        };

      % データ点
      \foreach \x/\y in {1/0.8,1.8/1.2,2.8/0.8,3.8/0.5}{
        \fill[blue] (\x,\y) circle (2.5pt);
      }

    \end{tikzpicture}

\end{frame}


\begin{frame}{考察}

\begin{itemize}
    \item したがって、構成比のグローバルな傾向(例えば、下のように、原則として産地に近い地域は比率が高く、産地から遠ざかれば比率がさがる)を表現する項と、今回の空間効果項を組み合わせることで、グローバルな線形の効果とローカルな非線形の効果を分離して推定することができるとより嬉しい。
\end{itemize}

    \begin{tikzpicture}[>=stealth,scale=1.3]
      % ---- 軸 ----
      \draw[->,thick] (-0.2,0) -- (6.6,0) node[below right]{産地からの距離};
      \draw[->,thick] (0,-0.2) -- (0,2.2) node[left]{比率};

      % y=1 目盛り
      \draw[-] (-0.06,1) -- (0.06,1);
      \node[left] at (0,1){1};

      % ---- 参照直線(黒の直線)----
      % 左上から右下へ単調減少の基準線
      \draw[thick] (0,1) -- (5.8,0.1);

      % ---- 減少カーブ(青の線)----
      % 手描き風のうねりを軽く入れた滑らかな曲線
      \draw[ultra thick,blue!80]
        plot[smooth] coordinates {
          (0.00,1)  % 左端:基準直線に沿う入り
          (1.00,0.80)  % 観測点
          (1.40,0.95)
          (1.80,1.20)  % 観測点
          (2.20,1.05)
          (2.50,0.90)
          (2.80,0.80)  % 観測点
          (3.20,0.65)
          (3.50,0.55)
          (3.80,0.50)  % 観測点
          (4.50,0.35)
          (5.30,0.20)
          (5.80,0.10)  % 右端:基準直線へ収束
        };

      % ---- 観測データ(前回と同じ座標)----
      % 必要に応じて座標だけ差し替えればOK
      \foreach \x/\y in {1/0.8,1.8/1.2,2.8/0.8,3.8/0.5}{
        \fill[blue] (\x,\y) circle (2.6pt);
      }
    \end{tikzpicture}

\end{frame}

\section{今後の展望}

\begin{frame}{今後の展望}

    今回のモデルを基礎に置くことで、色々と拡張の余地が生まれた:

    \begin{itemize}
        \item 線形効果・非線形効果の分離
        \item 共変量ごとの回帰係数の解釈・カーネルの選択、最適化
        \item データのsparsityを緩和するため、時間依存性の導入
    \end{itemize}

\end{frame}

\begin{frame}{今後の展望:時間依存性の導入}

Dynamic Gaussian Processの導入
\begin{itemize}
    \item 異なる時刻同士のGPの事前分布に、時間依存性を導入する。
    \item モデル化の自由度も高く、一方向or双方向や、単純なAR構造から状態空間モデルまで柔軟に選択できる
\end{itemize}

\begin{figure}
    \centering
    \includegraphics[width=1\linewidth]{fig/dynamicGP_1d.png}
    \caption{DGPの1次元の例(1方向AR)。過去の時刻の観測の影響が伝播している。}
\end{figure}

\begin{figure}
    \centering
    \includegraphics[width=0.75\linewidth]{fig/dynamicGP_2d.png}
    \caption{DGPの2次元の例。}
\end{figure}

\end{frame}

\begin{frame}{今後の展望:線形効果・非線形効果の分離}

産地からの距離に基づいて減衰する効果を
$$
\mu_k(s) = - \gamma_k \, d_k(s), \qquad k = 1,\ldots,K-1
$$

とする。 $d_k(s)$は地点$s$における産地$k$への距離(Tober's functionも使える)出あり、$\gamma_k > 0$ は距離に応じた減衰係数である。

これに加えて、
$$
\eta_k(s) = \mu_k(s) + \mathbf{W}(s)^\top \mathbf{b}_k(s) + \delta_k(s), \qquad k = 1,\ldots,K-1
$$

のように、線形効果と非線形効果$\delta_k(s)$(NNGP)の項を分けられる。


\end{frame}

\begin{frame}{その他取り組んだこと}

    NNGPを実装できたので、「遺跡の存在確率モデル」である点過程モデルに対するNHPP+NNGPモデルを(トイデータで)実装した

    \vspace{2mm}
    \begin{itemize}
        \item \hyperlink{https://www.notion.so/2509xx-Nearest-Neighbor-Gaussian-Process-2782a73a9fda80d39b28ce3cc4b2fa9b?source=copy_link}{Scalable inference for space-time Gaussian Cox processes}
    \end{itemize}

\end{frame}


\begin{frame}{まとめ}

\begin{itemize}
    \item 黒曜石の産地構成モデルとして、新たにGaussian Process事前分布を持つ多項ロジットモデルを提案し、実装した。
    \item 局所的な構成比の空間パターンを捉えることができた
    \item 今後のモデルの拡張の基礎となりそうなモデルを実装できた
\end{itemize}
    \begin{figure}
        \centering
        \includegraphics[width=0.5\linewidth]{fig/obsidian_nngp_multinomial/nngp_multinomial_origin1_period1.png}
    \end{figure}
\end{frame}
\begin{frame}
\frametitle{データとコード}

\url{https://github.com/ARUOHTA/bayesian_statistics}

\end{frame}

\bibliographystyle{unsrt} %参考文献出力スタイル
\bibliography{reference}

\end{document}
